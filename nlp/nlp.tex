\documentclass[a4paper,oneside,12pt]{ctexart}
\usepackage{enumerate,geometry,graphicx,float,setspace,bm,mathrsfs,xcolor,varwidth,framed,amsfonts,indentfirst,fancyhdr,BOONDOX-cal,tabularx,booktabs,colortbl}
\usepackage[colorlinks,linkcolor=red,anchorcolor=blue,citecolor=blue,urlcolor=blue]{hyperref}
\usepackage[thmmarks,hyperref]{ntheorem}
\usepackage{amsmath}
\usepackage{physics}
\usepackage{listings}
\usepackage{tikz}
\usepackage{asymptote}
\usepackage{cleveref}
\usepackage{titlesec}
\usepackage{fourier-otf}

\allowdisplaybreaks[4]
\onehalfspacing
\setlength{\headheight}{15pt}
\geometry{centering,left=2.54cm,right=2.54cm,top=3.18cm,bottom=3.18cm}
\newfontfamily\consolas{Consolas}
\definecolor{matlabgreen}{rgb}{0,0.5,0}
\definecolor{matlabpurple}{rgb}{0.75,0,0.75}
\lstset{
    language=C,
    basicstyle=\consolas,
    keywordstyle=\color{blue},
    commentstyle=\color{matlabgreen}\itshape,
    stringstyle=\color{matlabpurple}\ttfamily,
    frame=single,
    numbers=left,
    numberstyle=\tiny\consolas,
    breaklines,
    columns=flexible
}
\tikzset{every picture/.style={line width=0.75pt}} %set default line width to 0.75pt   

\crefname{exercise}{题目}{题目}
\crefname{figure}{图}{图}
\crefname{table}{表}{表}
\crefname{equation}{式}{式}
\crefdefaultlabelformat{(#2#1#3)}

\newcolumntype{C}{>{\centering\arraybackslash}X}

\titleformat{\section}[block]{\Large\bfseries}{\arabic{section}.}{0pt}{}[]

{
    \theoremstyle{plain}
    \theoremheaderfont{\normalfont\bfseries}
    \theorembodyfont{\kaishu}
    \theoremseparator{.}
    \newtheorem{exercise}{习题}
}

{
    \theoremstyle{nonumberplain}
    \theoremheaderfont{\bfseries}
    \theorembodyfont{\normalfont}
    \newtheorem{solution}{解答.}
}

\newcommand{\pr}{\mathbb{P}}
\newcommand{\E}{\mathbb{E}}
\renewcommand{\phi}{\varphi}
\renewcommand{\epsilon}{\varepsilon}
\renewcommand{\emptyset}{\varnothing}
\renewcommand{\liminf}{\varliminf}
\renewcommand{\limsup}{\varlimsup}

\begin{document}
    \begin{titlepage}
        \thispagestyle{fancy}
        \fancyfoot[C]{}
        \begin{center}
            \bfseries\zihao{-0}
            自然语言处理及应用
            
            实验报告
        \end{center}

        {
            \noindent\color[RGB]{200,22,30}\rule{\textwidth}{3pt}\\
            \rule[1.25em]{\textwidth}{0.75pt}
        }
        \vfill
        \begin{figure}[H]
            \centering
            \includegraphics[scale=0.4]{logo.png}
            \label{fig:logo}
        \end{figure}
        \vfill
        \begin{table}[H]
            \label{tab:报告人信息}
            \centering
            \LARGE
            \renewcommand{\arraystretch}{1.5}
            \setlength{\doublerulesep}{0pt}
            \arrayrulecolor[RGB]{200,22,30}
            \begin{tabularx}{\textwidth}{p{2cm}C}
                \textbf{名称} & \texttt{}\\
                \cmidrule{2-2}\morecmidrules\cmidrule{2-2}
                \textbf{姓名} & \texttt{孙思雨\quad 张卓立}\\
                \cmidrule{2-2}\morecmidrules\cmidrule{2-2}
                \textbf{班级} & \texttt{强基数学002\quad 强基数学001}\\
                \cmidrule{2-2}\morecmidrules\cmidrule{2-2}
                \textbf{学号} & \texttt{2206124483\quad 2204110786}\\
                \cmidrule{2-2}\morecmidrules\cmidrule{2-2}
                \textbf{日期} & \texttt{\today}\\
                \cmidrule{2-2}\morecmidrules\cmidrule{2-2}
            \end{tabularx}
        \end{table}
        \vfill
    \end{titlepage}
    \pagestyle{fancy}
    \fancyhead[C]{自然语言处理及应用实验报告}
    \fancyhead[L]{}
    \fancyhead[R]{}
    \section{实验要求}
 
    \section{实验环境}

    \section{实验思路}

    \section{实验结果}

    \section{实验总结}
\end{document}