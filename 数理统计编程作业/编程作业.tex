\documentclass[a4paper,oneside,12pt]{ctexart}
\usepackage{enumerate,geometry,graphicx,bm,mathrsfs,xcolor,varwidth,framed,amsfonts,amssymb,indentfirst,fancyhdr,listings}
\usepackage[colorlinks,linkcolor=red,anchorcolor=blue,citecolor=blue,urlcolor=blue]{hyperref}
\usepackage[thmmarks,hyperref]{ntheorem}
\usepackage{amsmath}
\usepackage{cleveref}

\setlength{\headheight}{15pt}
\newfontfamily\consolas{Consolas}
\allowdisplaybreaks[4]
\linespread{1.5}
\geometry{centering,left=2.54cm,right=2.54cm,top=3.18cm,bottom=3.18cm}
\pagestyle{fancy}
\fancyhead[L]{\kaishu 强基数学001}
\fancyhead[C]{\kaishu 张卓立}
\fancyhead[R]{\kaishu 2204110786}
\definecolor{matlabgreen}{rgb}{0,0.5,0}
\definecolor{matlabpurple}{rgb}{0.75,0,0.75}
\lstset{
    language=MATLAB,
    basicstyle=\consolas,
    keywordstyle=\color{blue},
    commentstyle=\color{matlabgreen}\itshape,
    stringstyle=\color{matlabpurple}\ttfamily,
    frame=single,
    numbers=left,
    numberstyle=\tiny\consolas
}

{
    \theoremstyle{plain}
    \theoremheaderfont{\normalfont\bfseries}
    \theorembodyfont{\kaishu}
    \theoremseparator{.}
    \newtheorem{exercise}{习题}
}

{
    \theoremstyle{nonumberplain}
    \theoremheaderfont{\bfseries}
    \theorembodyfont{\normalfont}
    \newtheorem{solution}{解.}
}

{
    \theoremstyle{nonumberplain}
    \theoremheaderfont{\bfseries}
    \theorembodyfont{\normalfont}
    \theoremsymbol{\ensuremath{\blacksquare}}
    \newtheorem{proof}{证明.}
}

\crefname{exercise}{习题}{习题}
\crefname{figure}{图}{图}
\crefname{table}{表}{表}
\crefname{equation}{式}{式}

\newcommand{\dif}{\mathrm{d}}
\newcommand{\differ}{\backslash}
\newcommand{\ptl}{\partial}
\newcommand{\R}{\mathbb{R}}
\newcommand{\N}{\mathbb{N}}
\newcommand{\C}{\mathbb{C}}
\newcommand{\Z}{\mathbb{Z}}
\renewcommand{\phi}{\varphi}
\renewcommand{\epsilon}{\varepsilon}
\newcommand{\abs}[1]{\left\vert#1\right\vert}
\newcommand{\norm}[1]{\left\Vert#1\right\Vert}
\newcommand{\expect}{\mathbb{E}}
\newcommand{\var}{\mathrm{Var}}
\newcommand{\prob}{\mathbb{P}}
\newcommand{\Exp}{\mathrm{Exp}}
\newcommand{\poi}{\mathrm{Poi}}
\newcommand{\Beta}{\mathrm{Beta}}

\begin{document}
    \begin{center}
        \bfseries\LARGE
        数理统计编程作业
    \end{center}

    \begin{exercise}
        \label{ex:1}
        假设$X_1,\cdots,X_n$是来自总体$X$的随机样本,$X\sim\chi^2(k)$.

        (1) 求样本均值$\bar{X}$的密度函数.

        (2) 求样本均值的渐进分布.

        (3) 通过编程比较,在不同样本量下,样本均值的密度函数和其渐进分布的密度函数图像.
    \end{exercise}

    \begin{exercise}
        \label{ex:2}
        在一个图上画出标准正态分布的密度曲线和$t(1),t(3),t(30),t(100)$的密度曲线.
    \end{exercise}

    \begin{exercise}
        \label{ex:3}
        令$X_1,\cdots,X_n$是来自均匀分布$U[\mu-\sqrt{3}\sigma,\mu+\sqrt{3}\sigma]$的随机样本,其中$-\infty <\mu<\infty,\sigma>0$.编程比较$\mu$
        的矩估计和极大似然估计的偏,方差和均方误差.
    \end{exercise}

\begin{lstlisting}
data=xlsread('历史预测数据输入.xlsx',4);
%输入所有待拟合的数据
X=data(:,1);
Y_K=data(:,2);
Y_Ca=data(:,3);
Y_Al=data(:,4);
Y_Fe=data(:,5);
%定义拟合的坐标范围和步长
x_min=63;
x_max=95;
x_int=0.01;
%记录所有拟合参数并且写入excel中
curve_fit=[];
%将拟合的参数写入excel
title= ['氧化钾(K2O)上界拟合  ';
        '氧化钾(K2O)下界拟合  ';
        '氧化钙(CaO)上界拟合  ';
        '氧化钙(CaO)下界拟合  ';
        '氧化铝(Al2O3)上界拟合';
        '氧化铝(Al2O3)下界拟合';
        '氧化铁(Fe2O3)上界拟合';
        '氧化铁(Fe2O3)下界拟合'];
title=cellstr(title);
xlswrite('高钾类玻璃包络线拟合结果.xlsx',title,1,'A1')
\end{lstlisting}
\end{document}