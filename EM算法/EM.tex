\documentclass[UTF8]{ctexbeamer}
\usepackage{float,enumerate,setspace,geometry,graphicx,bm,mathrsfs,xcolor,amsfonts,amssymb,indentfirst,fancyhdr,BOONDOX-cal}
\usepackage{amsmath}
\usepackage{physics}
\usepackage{subcaption}
\usepackage{algorithm2e}

\allowdisplaybreaks[4]

\newcommand{\differ}{\backslash}
\newcommand{\ptl}{\partial}
\newcommand{\R}{\mathbb{R}}
\newcommand{\N}{\mathbb{N}}
\newcommand{\C}{\mathbb{C}}
\newcommand{\Z}{\mathbb{Z}}
\renewcommand{\P}{\mathbb{P}}
\newcommand{\E}{\mathbb{E}}
\newcommand{\K}{\mathbb{K}}
\renewcommand{\i}{\mathrm{i}}
\renewcommand{\phi}{\varphi}
\renewcommand{\epsilon}{\varepsilon}
\renewcommand{\emptyset}{\varnothing}
\renewcommand{\liminf}{\varliminf}
\renewcommand{\limsup}{\varlimsup}

\usetheme{Madrid}
\usefonttheme{default}
\usecolortheme{whale}
\title{最大期望算法}
\author{吴天阳\and 张卓立}
\institute{XJTU\and 强基数学}
\date{2022年11月14日}

\begin{document}

        \maketitle
    
    \begin{frame}{收敛性}
        问题: 
        \begin{enumerate}
            \item EM算法是否收敛?
            \item 如果收敛, 能否收敛到全局最大值?
        \end{enumerate}
    \end{frame}

    \begin{frame}{收敛性}{问题1: EM算法是否收敛?}
        设有$m$个样本观察数据, $x=(x^{(1)},\cdots,x^{(m)})$, 
        模型参数$\theta$, 在观测数据中有未观察到的隐含数据$z=(z^{(1)},\cdots,z^{(m)})$, 
        令$Q_i\left(z^{(i)}\right)=\P\left(z^{(i)}\mid x^{(i)};\theta\right)$, 
        $\theta_j$表示根据EM算法迭代得到的参数$\theta$的各个估计值.

        要证明EM算法的收敛性, 只需证明对数似然的值在迭代过程中单调递增, 即 
        \begin{equation*}
            \sum_{i=1}^m\log \P\left(x^{(i)};\theta^{j+1}\right)\geqslant \sum_{i=1}^m\log\P\left(x^{(i)};\theta^j\right).
        \end{equation*}

        由于
        \begin{equation}
            \label{eq:L}
            L\left(\theta,\theta^j\right)=\sum_{i=1}^m\sum_{z^{(i)}}\P\left(z^{(i)}\mid x^{(i)};\theta^j\right)\log\P\left(x^{(i)},z^{(i)};\theta\right), 
        \end{equation}
        令
        \begin{equation}
            \label{eq:H}
            H(\theta,\theta^j)=\sum_{i=1}^m\sum_{z^{(i)}}\P\left(z^{(i)}\mid x^{(i)};\theta^j\right)\log\P\left(z^{(i)}\mid x^{(i)};\theta\right)
        \end{equation}
    \end{frame}

    \begin{frame}{收敛性}{问题1: EM算法是否收敛?}
        式$(\ref{eq:L})-(\ref{eq:H})$得到: 
        \begin{equation*}
            \sum_{i=1}^m\log\P\left(x^{(i)};\theta\right)=L(\theta,\theta^j)-H(\theta,\theta^j),
        \end{equation*}

        将$\theta^j,\theta^{j+1}$分别代入上式, 相减可得 
        \begin{align}
            &\sum_{i=1}^m\log\P\left(x^{(i)};\theta^{j+1}\right)-\sum_{i=1}^m\log\P\left(x^{(i)};\theta^j\right)\notag\\
            =&[L\left(\theta^{j+1},\theta^j\right)-L\left(\theta^j,\theta^j\right)]-[H\left(\theta^{j+1},\theta^j\right)-H\left(\theta^{j},\theta^j\right)].\label{eq:L-L,H-H}
        \end{align}
        下面证明式(\ref{eq:L-L,H-H})右端非负.
    \end{frame}

    \begin{frame}{收敛性}{问题1: EM算法是否收敛?}
        由于$\theta^{j+1}$使得$L(\theta,\theta^j)$极大, 所以 
        \begin{equation*}
            L\left(\theta^{j+1},\theta^j\right)-L\left(\theta^j,\theta^j\right)\geqslant 0,
        \end{equation*}
        又因为 
        \vspace*{-1pt}
        \begin{align}
            H\left(\theta^{j+1},\theta^j\right)-H\left(\theta^j,\theta^j\right)&=\sum_{i=1}^m\sum_{z^{(i)}}\P\left(z^{(i)}\mid x^{(i)};\theta^j\right)\log\frac{\P\left(z^{(i)}\mid x^{(i)};\theta^{j+1}\right)}{\P\left(z^{(i)}\mid x^{(i)};\theta^j\right)}\notag\\
            &\leqslant \sum_{i=1}^m\log\left[\sum_{z^{(i)}}\P\left(z^{(i)}\mid x^{(i)};\theta^{j+1}\right)\right.\notag\\
            &\quad\cdot \left.\frac{\P\left(z^{(i)}\mid x^{(i)};\theta^{j+1}\right)}{\P\left(z^{(i)}\mid x^{(i)};\theta^j\right)}\right]\label{eq:H-H jensen}\\
            &=\sum_{i=1}^m\log\sum_{z^{(i)}}\P\left(z^{(i)}\mid x^{(i)};\theta^{j+1}\right)=0,\notag
        \end{align}
    \end{frame}

    \begin{frame}{收敛性}{问题1: EM算法是否收敛?}
        其中式(\ref{eq:H-H jensen})用到Jensen不等式. 综上, 式(\ref{eq:L-L,H-H})得证, 即EM算法
        具有收敛性.
    \end{frame}

    \begin{frame}{收敛性}{问题2: 如果收敛, 能否收敛到全局最大值?}
        根据问题1的推导, EM算法可以保证收敛到一个稳定点, 但不能保证是
        全局最优的, 所以它是局部最优的算法. 

        若$L(\theta,\theta^j)$是凸的, 则可以收敛到全局最大值.
    \end{frame}
\end{document}