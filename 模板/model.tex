\documentclass[12pt,a4paper,oneside]{ctexart}
\usepackage{caption,graphicx}
\usepackage{amsmath}
\usepackage[thmmarks]{ntheorem}
\usepackage[centering]{geometry}
\usepackage{amsfonts,amssymb}
\usepackage{mathrsfs,extarrows}
\usepackage{bm}
\usepackage{pifont}
\usepackage{indentfirst}
\usepackage{setspace}
\usepackage{textcomp}
\usepackage[pages=some]{background}
\usepackage{xcolor, color}
\usepackage{colortbl,booktabs}
\usepackage{array,blkarray}
\usepackage{booktabs}
\usepackage{float}
\usepackage{fancyhdr}
\usepackage{tikz}
\usepackage{pgfplots}
\usepackage{enumerate}
\usepackage[colorlinks,linkcolor=red,anchorcolor=blue,citecolor=blue,urlcolor=blue]{hyperref}

\allowdisplaybreaks[4]
\linespread{1.5}
\geometry{left=2.54cm,right=2.54cm,top=3.18cm,bottom=3.18cm}
\usetikzlibrary{arrows}
\pgfplotsset{compat=1.15}
\definecolor{shadecolor}{RGB}{241, 241, 255}

{
    \theoremstyle{nonumberplain}
    \theoremheaderfont{\bfseries}
    \theorembodyfont{\normalfont}
    \theoremsymbol{\ensuremath{\blacksquare}}
    \newtheorem{proof}{证明.}
}
{
    \theoremstyle{plain}
    \theoremheaderfont{\normalfont\bfseries}
    \theorembodyfont{\kaishu}
    \newtheorem{theorem}{定理}[section]
}
{
    \theoremstyle{plain}
    \theoremheaderfont{\normalfont\bfseries}
    \theorembodyfont{\kaishu}
    \newtheorem{lemma}[theorem]{引理}
}
{
    \theoremstyle{nonumberplain}
    \theoremheaderfont{\normalfont\bfseries}
    \theorembodyfont{\ttfamily}
    \newtheorem{remark}{注.}
}
{
    \theoremstyle{plain}
    \theoremheaderfont{\bfseries}
    \theorembodyfont{\normalfont}
    \newtheorem{example}{例}[section]
}
{
    \theoremstyle{plain}
    \theoremheaderfont{\normalfont\bfseries}
    \theorembodyfont{\kaishu}
    \newtheorem{definition}[theorem]{定义}
}
{
    \theoremstyle{plain}
    \theoremheaderfont{\normalfont\bfseries}
    \theorembodyfont{\kaishu}
    \newtheorem{proposition}[theorem]{命题}
}
{
    \theoremstyle{plain}
    \theoremheaderfont{\normalfont\bfseries}
    \theorembodyfont{\kaishu}
    \newtheorem{corollary}[theorem]{推论}
}
\newcommand{\dif}{\mathrm{d}}
\newcommand{\differ}{\backslash}
\newcommand{\ptl}{\partial}
\newcommand{\R}{\mathbb{R}}
\newcommand{\N}{\mathbb{N}}
\renewcommand{\C}{\mathbb{C}}
\newcommand{\D}{\mathbb{D}}
\newcommand{\Z}{\mathbb{Z}}
\newcommand{\res}{\mathrm{Res}}
\renewcommand{\phi}{\varphi}
\renewcommand{\epsilon}{\varepsilon}
\newcommand{\abs}[1]{\left\vert#1\right\vert}

\begin{document}

    \begin{titlepage}
        \backgroundsetup{
            contents=\includegraphics{校徽.png},
            scale=0.5,
            placement=bottom,
            position=current page.center,
            opacity=0.1
        }
        \BgThispage
        \vspace*{0.3cm}
        \begin{center}
            \includegraphics[scale=0.6]{校名.png}
        \end{center}
        \vspace*{2.2cm}
        \begin{center}
            \Huge
            \bfseries
            这里是题目\\——模板
        \end{center}
        \vspace*{6.5cm}
        \begin{center}
            \large
            \begin{tabular}{rp{6cm}<{\centering}}
                \textbf{专业:}& \texttt{强基数学}\\
                \cline{2-2}
            \end{tabular}
        \end{center}
        \begin{center}
            \large
            \begin{tabular}{rp{6cm}<{\centering}}
                \textbf{学号:}& \texttt{2204110786}\\
                \cline{2-2}
            \end{tabular}
        \end{center}
        \begin{center}
            \large
            \begin{tabular}{rp{6cm}<{\centering}}
                \textbf{姓名:}& \texttt{张卓立}\\
                \cline{2-2}
            \end{tabular}
        \end{center}
    \end{titlepage}

    \newpage

    \tableofcontents

    \newpage

    \section{第一部分}
    这里是内容.
    \begin{theorem}
        这是定理.
    \end{theorem}

    \begin{lemma}
        这是引理.
    \end{lemma}

    \begin{proof}
        这是证明.
    \end{proof}
    
    \begin{remark}
        这是注.
    \end{remark}

    这是引用.\cite{薛庆超2019}

    \subsection{第一小部分}
    这里是内容.
    \subsubsection{第一小小部分}
    这里是内容.
    \subsubsection{第二小小部分}
    这里是内容.
    \subsection{第二小部分}
    这里是内容.
    \section{第二部分}
    这里是内容.

    \newpage
    
    \bibliographystyle{IEEEtran}
    \bibliography{model}
\end{document}