\documentclass[a4paper,oneside,12pt]{ctexart}
\usepackage{enumerate,geometry,graphicx,bm,mathrsfs,xcolor,varwidth,framed,amsfonts,amssymb,indentfirst,fancyhdr,BOONDOX-cal,frame}
\usepackage[colorlinks,linkcolor=red,anchorcolor=blue,citecolor=blue,urlcolor=blue]{hyperref}
\usepackage[thmmarks,hyperref]{ntheorem}
\usepackage{amsmath}
\usepackage{listings}

\setlength{\headheight}{15pt}
\allowdisplaybreaks[4]
\linespread{1.5}
\geometry{centering,left=2.54cm,right=2.54cm,top=3.18cm,bottom=3.18cm}
\pagestyle{fancy}
\fancyhead[L]{\kaishu 强基数学001}
\fancyhead[C]{\kaishu 张卓立}
\fancyhead[R]{\kaishu 2204110786}
\newfontfamily\consolas{Consolas}
\definecolor{matlabgreen}{rgb}{0,0.5,0}
\definecolor{matlabpurple}{rgb}{0.75,0,0.75}
\lstset{
    language=C,
    basicstyle=\consolas,
    keywordstyle=\color{blue},
    commentstyle=\color{matlabgreen}\itshape,
    stringstyle=\color{matlabpurple}\ttfamily,
    frame=single,
    numbers=left,
    numberstyle=\tiny\consolas
}

{
    \theoremstyle{plain}
    \theoremheaderfont{\normalfont\bfseries}
    \theorembodyfont{\kaishu}
    \newtheorem{exercise}{习题}
}

{
    \theoremstyle{nonumberplain}
    \theoremheaderfont{\bfseries}
    \theorembodyfont{\normalfont}
    \newtheorem{solution}{解.}
}

{
    \theoremstyle{nonumberplain}
    \theoremheaderfont{\bfseries}
    \theorembodyfont{\normalfont}
    \theoremsymbol{\ensuremath{\blacksquare}}
    \newtheorem{proof}{证明.}
}

\renewcommand{\phi}{\varphi}
\renewcommand{\epsilon}{\varepsilon}
\renewcommand{\emptyset}{\varnothing}
\newcommand{\abs}[1]{\left\vert#1\right\vert}
\newcommand{\norm}[1]{\left\Vert#1\right\Vert}
\renewcommand{\liminf}{\varliminf}
\renewcommand{\limsup}{\varlimsup}

\begin{document}
    \begin{center}
        \bfseries\LARGE
        算法设计与分析作业
    \end{center}

    \begin{exercise}
        \label{ex:2-7}
        设$P(x)=a_0+a_1x+\cdots+a_dx^d$是一个$d$次多项式.假设已有一算法能在$O(i)$时间内计算一个$i$次多项式与一个1次多项式的乘积,
        以及一个算法能在$O(i\log i)$时间内计算两个$i$次多项式的乘积,对于任意给定的$d$个整数$n_1,n_2,\cdots,n_d$,用分治法设计一个有效算法,
        计算出$P(n_1)=P(n_2)=\cdots=P(n_d)=0$且最高次项系数为1的$d$次多项式$P(x)$,并分析算法的效率.
    \end{exercise}

    \begin{solution}
        将$P(x)$写为 
        \begin{equation*}
            P(x)=\prod_{i=1}^d(x-n_i)=\prod_{i=1}^{\lfloor d/2\rfloor}(x-n_i)\prod_{i=\lfloor d/2\rfloor+1}^d(x-n_i)=P_1(x)P_2(x)
        \end{equation*}
        用分治法将$d$次多项式转化为两个$d/2$次多项式的乘积.

        设这个算法计算$d$次多项式所需时间为$T(d)$,那么$T(d)$满足 
        \begin{equation*}
            T(d)=\begin{cases}
                O(1) & d=1\\
                2T(d/2)+O(d\log d) & d>1
            \end{cases}
        \end{equation*}
        那么$T(d)=O(d\log^2 d)$.
    \end{solution}

    \begin{exercise}
        \label{ex:2-9}
        设$T[0:n-1]$是$n$个元素的数组.对任一元素$x$,设$S(x)=\{i\mid T[i]=x\}$.当$\abs{S(x)}>n/2$时,称$x$为$T$的主元素.设计一个线性
        时间算法,确定$T[0:n-1]$是否有一个主元素.
    \end{exercise}

    \begin{solution}
        如果$T$有主元素$x$,那么$x$是$T$的中位数,即若$T$的中位数不是主元素,那么$T$无主元素.

        那么用线性时间寻找中位数算法可以在线性时间内判定$T$是否有主元素.
    \end{solution}

    \begin{exercise}
        \label{ex:2-10}
        若在习题2-9中,数组$T$中的元素不存在序关系,只能测试任意两个元素是否相等,试设计一个有效算法确定$T$是否有一个主元素.算法的计算复杂性应为$O(n\log n)$.
        更进一步,能找到一个线性时间算法吗?
    \end{exercise}

    \begin{solution}
        用分治法寻找$T[1:n]$的主元素,设$x$是主元素,那么$S_x=\{i\mid T[i]=x\}$,则$\abs{S_x}>n/2$.将$T[1:n]$分为数组$T[1:n/2]$和$T[(n/2+1:n]$,
        那么$x$是$T[1:n/2]$的主元素或者$T[n/2+1:n]$的主元素,此时只需对两个数组线性扫描,那么 
        \begin{equation*}
            T(n)=\begin{cases}
                O(1)&n\leqslant 4\\
                2T(n/2)+O(n)&n>4
            \end{cases}
        \end{equation*}
        则$T(n)=O(n\log n)$.

        用下面的算法可以在$O(n)$时间内找出$T[1:n]$的主元素.

        对$1\leqslant i\leqslant n/2$,当$T[2i-1]=T[2i]$,将$T[2i]$村存入数组$A$,否则不进行任何操作.

        那么$A$中元素个数$\abs{A}\leqslant n/2$,若$x$是$T[1:n]$的主元素,那么$x$是$A$的主元素或者$T[n]$是$T[1:n]$的主元素.

        程序:
\begin{lstlisting}
#include <iostream>
#include <stdio.h>
#include <stdlib.h>
#include <math.h>

using namespace std;
#define N 10

int Partition(int* T, int l, int r)
{
	if (l == r)
	{
		return T[l];
	}

	int m = (l + r) / 2;
	int n1 = Partition(T, l, m);
	int n2 = Partition(T, m + 1, r);

	if (n1 == n2)
		return n1;
	else
	{
		int cnt1 = 0, cnt2 = 0;
		for (int i = l; i <= r; i++)
		{
			if (T[i] == n1)
				cnt1++;
			if (T[i] == n2)
				cnt2++;
		}
		int half = (l + r) / 2;
		if (cnt1 > half)
			return n1;
		else if (cnt2 > half)
			return n2;
		else
			return INT_MAX;
	}
}

int main()
{
	int T[N] = { 5,5,5,5,5,5,1,3,4,6 };
	cout << "数组T的元素如下:" << endl
	for (int i = 0; i < N; i++)
	{
		cout << T[i] << " ";
	}
	cout << endl;
	if (Partition(T, 0, N - 1) != INT_MAX)
		cout << "该数组的主元素为:" ;
        << Partition(T, 0, N - 1) << endl;
	else
		cout << "该数组没有主元素" << endl;
	return 0;
}
\end{lstlisting}
结果:
    \begin{lstlisting}
数组T的元素如下:
5 5 5 5 5 5 1 3 4 6
该数组的主元素为:5
    \end{lstlisting}
    \end{solution}

    \begin{exercise}
        \label{ex:2-28}
        设$X[0:n-1]$和$Y[0:n-1]$为两个数组,每个数组中含有$n$个已排好序的数.试设计$O(\log n)$时间的算法,找出$X$和$Y$的$2n$个数的中位数.
    \end{exercise}

    \begin{solution}
        设$X[i_1:j_1]$和$Y[i_2:j_2]$是$X$和$Y$的排好序的子数组,且$j_1-i_1=j_2-i_2$.
        
        若$X[i_1]\leqslant Y[j_2]$,则中位数$m$满足$X[i_1]\leqslant m\leqslant Y[j_2]$,若$X[i_1]\geqslant Y[j_2]$,则$Y[j_2]\leqslant m\leqslant X[i_1]$.

        设$m_1=(i_1+j_2)/2$,$m_2=(i_2+j_2)/2$.那么$m_1+m_2=i_1+j_2$.

        当$X[m_1]=Y[m_2]$时,$m=X[m_1]=Y[m_2]$.

        当$X[m_1]<Y[m_2]$时,设$m_1$是$X[m_1:j_1]$和$Y[j_2:m_2]$的中位数,则$m=m_1$.

        当$X[m_1]>Y[m_2]$时,设$m_2$是$X{i_1:m_1}$和$Y[m_2:j_2]$的中位数,类似地有$m=m_2$.

        可得时间复杂度是$O(\log n)$.

        程序:
        \begin{lstlisting}
#include <iostream>
using namespace std;
void find(int a[], int l1, int r1, int b[], int l2, int r2)
{
	int m1, m2;
	if (r1 - l1 == 1) {
		m1 = a[l1] < b[l2] ? b[l2] : a[l1];
		m2 = a[r1] < b[r2] ? a[r1] : b[r2];
		if (m1 < m2)cout << m1 << endl;
		else cout << m2 << endl;
		return;
	}
	if (r1 == l1)
	{
		int k = a[r1] < b[r2] ? a[r1] : b[r2];
		cout << k << endl;
		return;
	}
	int mid1 = (l1 + r1) / 2, mid2 = (l2 + r2 + 1) / 2;
	if (a[mid1] == b[mid2])
	{
		cout << a[mid1] << endl;
		return;
	}
	if (a[mid1] < b[mid2])find(a, mid1, r1, b, l2, mid2);
	if (a[mid1] > b[mid2])find(a, l1, mid1, b, mid2, r2);
}
int main()
{
	int a[100001], b[100001];
	int n;
	cout << "输入n" << endl;
	cin >> n;
	cout << "输入a" << endl;
	for (int i = 0; i < n; i++)
	{
		cin >> a[i];
	}
	cout << "输入b" << endl;
	for (int i = 0; i < n; i++)
	{
		cin >> b[i];
	}
	cout << "结果" << endl;
	find(a, 0, n - 1, b, 0, n - 1);
}
        \end{lstlisting}
        结果:
        \begin{lstlisting}
输入n
5
输入a
1 6 8 3 4
输入b
9 4 6 8 5
结果
6
        \end{lstlisting}
    \end{solution}
\end{document}