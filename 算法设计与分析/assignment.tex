\documentclass[a4paper,oneside,12pt]{ctexart}
\usepackage{enumerate,geometry,graphicx,float,setspace,bm,mathrsfs,xcolor,varwidth,framed,amsfonts,amssymb,indentfirst,fancyhdr,BOONDOX-cal,frame}
\usepackage[colorlinks,linkcolor=red,anchorcolor=blue,citecolor=blue,urlcolor=blue]{hyperref}
\usepackage[thmmarks,hyperref]{ntheorem}
\usepackage{amsmath}
\usepackage{physics}
\usepackage{listings}
\usepackage{minted}
\usepackage{tikz}
\usepackage{asymptote}
\usepackage{cleveref}

\setlength{\headheight}{15pt}
\allowdisplaybreaks[4]
\onehalfspacing
\geometry{centering,left=2.54cm,right=2.54cm,top=3.18cm,bottom=3.18cm}
\pagestyle{fancy}
\fancyhead[L]{\kaishu 强基数学001}
\fancyhead[C]{\kaishu 张卓立}
\fancyhead[R]{\kaishu 2204110786}

\newminted{c}{breaklines,linenos,autogobble,frame=lines,style=xcode}
\newmintinline{c}{breaklines,style=xcode}

\lstloadlanguages{[Visual]C++}
\newfontfamily\consolas{Consolas}
\definecolor{matlabgreen}{rgb}{0,0.5,0}
\definecolor{matlabpurple}{rgb}{0.75,0,0.75}
\lstset{
    language=[Visual]C++,
    basicstyle=\consolas,
    keywordstyle=\color{blue},
    commentstyle=\color{matlabgreen}\itshape,
    stringstyle=\color{matlabpurple}\ttfamily,
    frame=tb,
    numbers=left,
    numberstyle=\tiny\consolas,
    breaklines,
    columns=flexible,
    morecomment=[s][\color{matlabgreen}]{/*}{*/},
    emph={endif},
    emphstyle=\color{blue}
}
\tikzset{every picture/.style={line width=0.75pt}} %set default line width to 0.75pt   

\crefname{exercise}{习题}{习题}
\crefname{figure}{图}{图}
\crefname{table}{表}{表}
\crefname{equation}{式}{式}
\crefdefaultlabelformat{(#2#1#3)}

{
    \theoremstyle{plain}
    \theoremheaderfont{\normalfont\bfseries}
    \theorembodyfont{\kaishu}
    \theoremseparator{.}
    \newtheorem{exercise}{习题}
}

{
    \theoremstyle{nonumberplain}
    \theoremheaderfont{\bfseries}
    \theorembodyfont{\normalfont}
    \newtheorem{solution}{解.}
}

{
    \theoremstyle{nonumberplain}
    \theoremheaderfont{\bfseries}
    \theorembodyfont{\normalfont}
    \theoremsymbol{\ensuremath{\blacksquare}}
    \newtheorem{proof}{证明.}
}

\renewcommand{\phi}{\varphi}
\renewcommand{\epsilon}{\varepsilon}
\renewcommand{\emptyset}{\varnothing}
\renewcommand{\liminf}{\varliminf}
\renewcommand{\limsup}{\varlimsup}

\begin{document}
    \begin{center}
        \bfseries\LARGE
        算法设计与分析(第五章)作业
    \end{center}

    \begin{exercise}[5-4]
        \label{ex:5.4}
        试设计一个解最大团问题的迭代回溯法.
    \end{exercise}

    \begin{solution}
        图$G$的最大团问题可以看作是$G$的顶点集$V$的子集选取问题, 因此可以用子集树表示问题的解空间.

        $x[i]$用来当前解,
        \begin{equation*}
           x[i]=\begin{cases}
            1& \text{结点$i$在当前解中},\\
            0 & \text{$i$不在当前解中},
        \end{cases}
        \end{equation*}
        $n$表示图$G$的顶点数, $cn$表示当前顶点数, $bestn$表示当前最大顶点数, 
        $bestx[i]$用来记录最优解,
        \begin{equation*}
            bestx[i]=\begin{cases}
                1& \text{结点$i$在最优解中},\\
                0 & \text{$i$不在最优解中},
            \end{cases}
        \end{equation*}
        $bestn$表示当前最大顶点数, $a[i][j]$是$G$的邻接矩阵.

        首先将$x[i]$初始化, 令$\forall i,x[i]=0$. 若对任意的已经在团中的结点$j<i$, $(i,j)\notin E$, 即$x[j]>0$且$a[i][j]=0$, 
        则$i$可以加入团中, 反之不能.

        之后采用树的非递归深度优先遍历算法, 可以将回溯法表示为一个非递归迭代过程.
    \end{solution}

    \begin{exercise}[5-6]
        \label{ex:5-6}
        设$G$是有$n$个顶点的有向图, 从顶点$i$发出的边的最小费用记为$\min(i)$.

        \begin{enumerate}[(1)]
            \item 证明图$G$的所有前缀为$x[1:i]$的旅行售货员回路的费用至少为$\sum_{j=2}^ia(x_{j-1},x_j)+\sum_{j=i}^n\min(x_j)$, 其中$a(u,v)$是边$(u,v)$的费用.
            \item 利用上述结论设计一个高效的上界函数, 重写旅行售货员问题的回溯法, 并与教材中的算法进行比较.
        \end{enumerate}
    \end{exercise}

    \begin{solution}
        (1) 前缀为$x[i:i]$的旅行售货员回路任一旅行售货员回路可表示为$(x[1],\cdots,x[i],\\\pi(i+1),\cdots,\pi(n))$.

        费用为
        \begin{equation*}
            h(\pi)=\sum_{j=2}^ia(x_{j-1},x_j)+a(x_i,\pi(i+1))+\sum_{j=i+1}^na(\pi(j),\pi(j\bmod n+1)),
        \end{equation*}
        那么 
        \begin{align*}
            h(\pi)&\geqslant \sum_{j=2}^ia(x_{j-1},x_j)+\min(x_i)+\sum_{j=i+1}^n\min(\pi(j))\\
            &=\sum_{j=2}^ia(x_{j-1},x_j)+\sum_{j=i}^n\min(x_j)
        \end{align*}

        (2) 对图$G$进行简单遍历, 计算出$\sum_{i=1}^n\min(i)$.
    \end{solution}
\end{document}