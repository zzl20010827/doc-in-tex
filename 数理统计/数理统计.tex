\documentclass[a4paper,oneside,12pt]{ctexart}
\usepackage{enumerate,geometry,graphicx,bm,mathrsfs,xcolor,varwidth,framed,amsfonts,amssymb,indentfirst,fancyhdr}
\usepackage[colorlinks,linkcolor=red,anchorcolor=blue,citecolor=blue,urlcolor=blue]{hyperref}
\usepackage[thmmarks,hyperref]{ntheorem}
\usepackage{amsmath}
\usepackage{cleveref}

\setlength{\headheight}{15pt}
\allowdisplaybreaks[4]
\linespread{1.5}
\geometry{centering,left=2.54cm,right=2.54cm,top=3.18cm,bottom=3.18cm}
\pagestyle{fancy}
\fancyhead[L]{\kaishu 强基数学001}
\fancyhead[C]{\kaishu 张卓立}
\fancyhead[R]{\kaishu 2204110786}

{
    \theoremstyle{plain}
    \theoremheaderfont{\normalfont\bfseries}
    \theorembodyfont{\kaishu}
    \newtheorem{exercise}{习题}
}

{
    \theoremstyle{nonumberplain}
    \theoremheaderfont{\bfseries}
    \theorembodyfont{\normalfont}
    \newtheorem{solution}{解.}
}

{
    \theoremstyle{nonumberplain}
    \theoremheaderfont{\bfseries}
    \theorembodyfont{\normalfont}
    \theoremsymbol{\ensuremath{\blacksquare}}
    \newtheorem{proof}{证明.}
}

\crefname{exercise}{习题}{习题}
\crefname{figure}{图}{图}
\crefname{table}{表}{表}
\crefname{equation}{式}{式}

\newcommand{\dif}{\mathrm{d}}
\newcommand{\differ}{\backslash}
\newcommand{\ptl}{\partial}
\newcommand{\R}{\mathbb{R}}
\newcommand{\N}{\mathbb{N}}
\newcommand{\C}{\mathbb{C}}
\newcommand{\Z}{\mathbb{Z}}
\renewcommand{\phi}{\varphi}
\renewcommand{\epsilon}{\varepsilon}
\newcommand{\abs}[1]{\left\vert#1\right\vert}
\newcommand{\norm}[1]{\left\Vert#1\right\Vert}
\newcommand{\expect}{\mathbb{E}}
\newcommand{\var}{\mathrm{Var}}
\newcommand{\prob}{\mathbb{P}}
\newcommand{\Exp}{\mathrm{Exp}}
\newcommand{\poi}{\mathrm{Poi}}
\newcommand{\Beta}{\mathrm{Beta}}

\begin{document}
    \begin{center}
        \LARGE\bfseries
        数理统计作业
    \end{center}

    \begin{exercise}
        \label{ex:1}
        设$X_1,\cdots.X_n$是来自$N(0,1)$的随机样本,令 
    \begin{equation*}
        \bar{X}_k=\frac{1}{k}\sum_{i=1}^kX_i,\quad \bar{X}_{n-k}=\frac{1}{n-k}\sum_{i=k+1}^nX_i.
    \end{equation*}
    求(1)-(4)的分布.(1) $\frac{1}{2}(\bar{X}_k+\bar{X}_{n-k})$.(2) $k\bar{X}^2_k+(n-k)\bar{X}_{n-k}^2$.(3) $X_1^2/X_2^2$.(4) 
    $X_1/X_n$.
    \end{exercise}

    \begin{solution}
        (1) 因为$X_i\sim N(0,1)$且相互独立,则$\bar{X}_k\sim N(0,1/k),\bar{X}_{n-k}\sim N(0,1/(n-k))$,那么$\frac{1}{2}(\bar{X}_k+\bar{X}_{n-k})\sim N\left(0,\frac{n}{4k(n-k)}\right)$.

        (2) $(n-k)\bar{X}_{n-k}^2=\left(\sqrt{n-k}\bar{X}_{n-k}\right)^2$,因为$\sqrt{n-k}\bar{X}_{n-k}\sim N(0,1)$,那么
        \begin{equation*}
            \left(\sqrt{n-k}\bar{X}_{n-k}\right)^2\sim \chi^2(1).
        \end{equation*}
        同理,$k\bar{X}_k^2\sim \chi^2(1)$,那么$k\bar{X}_k^2+(n-k)\bar{X}_{n-k}^2\sim \chi^2(2)$.

        (3) $X_1^2\sim\chi^2(1)$且$X_2^2\sim\chi^2(1)$,则
        \begin{equation*}
            \frac{X_1^2}{X_2^2}\sim F(1,1).
        \end{equation*}

        (4) 设$X_1/X_n$的分布函数为$F(x)$,$X_1,X_n$的联合概率密度函数为$p(x_1,x_n)$,那么
        \begin{align*}
            F(x)&=\prob\left(\frac{X_1}{X_n}<x\right)=\prob\left(\frac{X_1}{X_n}<x,X_n>0\right)+\prob\left(\frac{X_1}{X_n}<x,X_n<0\right)\\
            &=\int_0^{+\infty}\dif x_n\int_{-\infty}^{xx_n}p(x_1,x_n)\dif x_1+\int_{-\infty}^0\dif x_n\int_{xx_n}^{+\infty}p(x_1,x_n)\dif x_1.
        \end{align*}
        $X_1,X_n$为独立的标准正态分布随机变量,那么$X_1/X_n$的概率密度函数为 
        \begin{equation*}
            f(x)=\frac{\dif F(x)}{\dif x}=\frac{1}{x^2}\sqrt{\frac{2}{\pi}}.
        \end{equation*}
    \end{solution}

    \begin{exercise}
        \label{ex:2}
        设$X_1,\cdots,X_n$是来自双参数指数分布 
        \begin{equation*}
            f(x,\mu,\sigma)=\frac{1}{\sigma}\exp\left\{-\frac{x-\mu}{\sigma}\right\},\quad x\geqslant \mu
        \end{equation*}
        的简单随机样本,其中$-\infty<\mu<\infty,\sigma>0$,求$\mu,\sigma$和$\prob(X_1\geqslant t)(t>\mu)$的矩估计和极大似然估计.
    \end{exercise}

    \begin{solution}
        令$r$是正整数,那么 
        \begin{align*}
            \expect(X^r)&=\int_\mu^{+\infty}x^r\frac{1}{\sigma}\exp\left\{-\frac{x-\mu}{\sigma}\right\}\\
            &=\int_0^{+\infty}(x+\mu)^r\frac{1}{\sigma}\exp\left\{-\frac{x}{\sigma}\right\}\\
            &=\sum_{k=0}^r\left(\begin{array}{c}
                r\\
                k
            \end{array}\right)\mu^{r-k}\sigma^k\int_0^{+\infty}\left(\frac{x}{\sigma}\right)^k\exp\left(-\frac{x}{\sigma}\right)\dif \frac{x}{\sigma}\\
            &=\sum_{k=0}^r\left(\begin{array}{c}
                r\\
                k
            \end{array}\right)\mu^{r-k}\sigma^k\Gamma(k+1)\\
            &=r!\sum_{k=0}^r\frac{\mu^{r-k}\sigma^k}{(r-k)!}.
        \end{align*}
        那么要求$\mu,\sigma^2$的矩估计,即求 
        \begin{equation*}
            \begin{cases}
                \frac{X_1+\cdots+X_n}{n}=\mu+\sigma\\
                \frac{X_1^2+\cdots+X_n^2}{n}=\mu^2+2\mu\sigma+2\sigma^2
            \end{cases}
        \end{equation*}
        的解,解为 
        \begin{equation*}
            \begin{cases}
                \hat{\sigma}=\sqrt{\frac{X_1^2+\cdots+X_n^2}{n}-\frac{(X_1+\cdots+X_n^2)^2}{n^2}}\\
                \hat{\mu}=\frac{X_1+\cdots+X_n}{n}-\sqrt{\frac{X_1^2+\cdots+X_n^2}{n}-\frac{(X_1+\cdots+X_n^2)^2}{n^2}}
            \end{cases}
        \end{equation*}

        又因为
        \begin{align*}
            \prob(X_1\geqslant t)&=\int_t^{+\infty}\frac{1}{\sigma}\exp\left(-\frac{x-\mu}{\sigma}\right)\dif x\\
            &=\exp\left(-\frac{t-\mu}{\sigma}\right).
        \end{align*}
        那么$\prob(X_1\geqslant t)$的矩估计为 
        \begin{equation*}
            \widehat{\prob(X_1\geqslant t)}=\exp\left(-\frac{t-\frac{X_1+\cdots+X_n}{n}+\sqrt{\frac{X_1^2+\cdots+X_n^2}{n}-\frac{(X_1+\cdots+X_n^2)^2}{n^2}}}{\sqrt{\frac{X_1^2+\cdots+X_n^2}{n}-\frac{(X_1+\cdots+X_n^2)^2}{n^2}}}\right)
        \end{equation*}

        $X_1,\cdots,X_n$的联合概率密度为 
        \begin{equation*}
            L(\mu,\sigma)=\prod_{k=1}^nf(x_k,\mu,\sigma)=\prod_{k=1}^n\left(\frac{1}{\sigma}\exp\left(-\frac{x-\mu}{\sigma}\right)\right).
        \end{equation*}
        取对数,令 
        \begin{equation*}
            I(\mu,\sigma)=\log L(\mu,\sigma)=\sum_{k=1}^n\left(-\log\sigma-\frac{x_k-\mu}{\sigma}\right).
        \end{equation*}
        求一阶导和二阶导,
        \begin{gather*}
            \nabla I=\left(\frac{n}{\sigma},-\frac{n}{\sigma}+\frac{\sum_{k=1}^nx_k}{\sigma^2}+\frac{n\mu}{\sigma}\right),\\
            \nabla^2 I=\begin{bmatrix}
                0 & -\frac{n}{\sigma^2}\\
                -\frac{n}{\sigma^2} & \frac{n}{\sigma^2}-\frac{2\sum_{k=1}^nx_k}{\sigma^3}+\frac{n\mu}{\sigma^3}.
            \end{bmatrix}
        \end{gather*}
        得到$\sigma=+\infty$时取最大.极大似然估计:
        \begin{equation*}
            \hat{\mu}\in\R,\quad \hat{\sigma}=+\infty,\quad \widehat{\prob(X_1\geqslant t)}=1.
        \end{equation*}
    \end{solution}

    \begin{exercise}
        \label{ex:3}
        设$X_1,\cdots,X_n$和$Y_1,\cdots,Y_n$分别为来自正态分布总体$N(\mu_1,\sigma^2)$和$N(\mu_2,\sigma^2)$的随机样本,求$\mu_1,\mu_2,\sigma^2$的MLE.
    \end{exercise}

    \begin{solution}
        $X_1,\cdots,X_n$的累计概率密度为 
        \begin{equation*}
            \prod_{i=1}^n\frac{1}{\sqrt{2\pi}\sigma}\exp\left(-\frac{(x_i-\mu_1)^2}{2\sigma^2}\right).
        \end{equation*}
        取对数,关于$\mu,\sigma$求偏导,令偏导为0得到方程
        \begin{equation*}
            \begin{cases}
                n\mu_1=\sum_{i=1}^nx_i\\
                -n\sigma^2+\sum_{i=1}^n(x_i-\mu_1)^2=0.
            \end{cases}
        \end{equation*}
        解,得 
        \begin{equation*}
            \begin{cases}
                \hat{\mu}_1=\frac{1}{n}\sum_{i=1}^nx_i\\
                \hat{\sigma}^2=\frac{1}{n}\sum_{i=1}^n\left(x_i-\frac{1}{n}\sum_{i=1}^nx_i\right)^2.
            \end{cases}
        \end{equation*}

        同理可得 
        \begin{equation*}
            \begin{cases}
                \hat{\mu}_2=\frac{1}{n}\sum_{i=1}^ny_i\\
                \hat{\sigma}^2=\frac{1}{n}\sum_{i=1}^n\left(y_i-\frac{1}{n}\sum_{i=1}^ny_i\right)^2.
            \end{cases}
        \end{equation*}
    \end{solution}

    \begin{exercise}
        \label{ex:4}
        设$X_1,\cdots,X_n$来自均匀分布$U(\theta,2\theta)$的随机样本,其中$\theta\in(0,\infty)$,求:
        (1) $\theta$的极大似然估计$\hat{\theta}$;(2) 判断$\hat{\theta}$是否为无偏估计,如果不是无偏估计,基于$\hat{\theta}$构造一个无偏估计.
    \end{exercise}

    \begin{solution}
        (1) $X_i$的概率密度为 
        \begin{equation*}
            p(x)=\begin{cases}
                \frac{1}{\theta}, &\theta<x<\theta\\
                0,& x\leqslant\theta\text{或}x\geqslant\theta.
            \end{cases}
        \end{equation*}
        那么当$(x_1,\cdots,x_n)\in(\theta,2\theta)^n$时,$X_1,\cdots,X_n$的联合密度函数为 
        \begin{equation*}
            f(x_1,\cdots,x_n,\theta)=\frac{1}{\theta^n},
        \end{equation*}
        其余情况下$f(x_1,\cdots,x_n,\theta)=0$.当$(x_1,\cdots,x_n)\in(\theta,2\theta)$时,$\frac{1}{2}\max_ix_i\leqslant\theta\leqslant\min_ix_i$.
        那么$\theta$的极大似然估计为$\hat{\theta}=\frac{1}{2}X_{(n)}$.

        (2) $X_{(n)}$的分布函数为 
        \begin{equation*}
                \prob(X_{(n)}<x)=\prod_{i=1}^n\prob(X_i<x)=F(x)^n
        \end{equation*}
        其中$F(x)$是$X_i$的分布函数,那么可以得到 
        \begin{align*}
            \expect(\hat{\theta})&=\frac{1}{2}\int_\theta^{2\theta}\frac{nx(x-\theta)^{n-1}}{\theta^n}\dif x\\
            &=\frac{2n+1}{2n+2}\theta\neq\theta.
        \end{align*}
        则它不是无偏估计.

        根据上面的期望,可知$\frac{n+1}{2n+1}X_{(n)}$是一个无偏估计.
    \end{solution}

    \begin{exercise}
        \label{ex:5}
        设$X$服从对数正态分布,即$\log X\sim N(\mu,\sigma^2)$,其中$\mu\in \R,0<\sigma<+\infty$.设$X_1,\cdots,X_n$来自总体$X$的随机样本,
        求:(1) $\mu,\sigma^2$的极大似然估计$\hat{\theta}$;(2) 求$\expect(X)$的极大似然估计. 
    \end{exercise}

    \begin{solution}
        (1) $\log X_1,\cdots,\log X_n$是$\log X$的随机样本,根据\cref{ex:3}的结果,可得极大似然估计 
        \begin{equation*}
            \begin{cases}
                \hat{\mu}=\frac{1}{n}\sum_{i=1}^n\log X_i\\
                \hat{\sigma}^2=\frac{1}{n}\sum_{i=1}^n\left(\log X_i-\sum_{i=1}^n\log X_i\right)^2.
            \end{cases}
        \end{equation*}

        (2) 注意到$X=e^{\log X}$,那么 
        \begin{align*}
            \expect(X)&=\expect(e^{\log X})\\
            &=\int_{-\infty}^\infty\exp(x)\frac{1}{\sqrt{2\pi}\sigma}\exp\left(-\frac{(x-\mu)^2}{2\sigma^2}\right)\dif x\\
            &=\exp\left(\frac{\sigma^2+2\mu}{2}\right)\int_{-\infty}^\infty\frac{1}{\sqrt{2\pi}\sigma}\exp\left(-\frac{(x-(\sigma^2+\mu))^2}{2\sigma^2}\right)\dif x\\
            &=\exp\left(\frac{\sigma^2+2\mu}{2}\right).
        \end{align*}
        代入第(1)问的极大似然估计,得到 
        \begin{equation*}
            \hat{\expect}(X)=\exp\left(\frac{1}{2n}\sum_{i=1}^n\left(\log X_i-\sum_{i=1}^n\log X_i\right)^2+\frac{1}{n}\sum_{i=1}^n\log X_i\right).
        \end{equation*}
    \end{solution}
\end{document}