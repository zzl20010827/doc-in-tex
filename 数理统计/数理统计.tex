\documentclass[a4paper,oneside,12pt]{ctexart}
\usepackage{enumerate,geometry,graphicx,bm,mathrsfs,xcolor,varwidth,framed,amsfonts,amssymb,indentfirst,fancyhdr}
\usepackage[colorlinks,linkcolor=red,anchorcolor=blue,citecolor=blue,urlcolor=blue]{hyperref}
\usepackage[thmmarks,hyperref]{ntheorem}
\usepackage{amsmath}
\usepackage{cleveref}

\setlength{\headheight}{15pt}
\allowdisplaybreaks[4]
\linespread{1.5}
\geometry{centering,left=2.54cm,right=2.54cm,top=3.18cm,bottom=3.18cm}
\pagestyle{fancy}
\fancyhead[L]{\kaishu 强基数学001}
\fancyhead[C]{\kaishu 张卓立}
\fancyhead[R]{\kaishu 2204110786}

{
    \theoremstyle{plain}
    \theoremheaderfont{\normalfont\bfseries}
    \theorembodyfont{\kaishu}
    \newtheorem{exercise}{习题}
}

{
    \theoremstyle{nonumberplain}
    \theoremheaderfont{\bfseries}
    \theorembodyfont{\normalfont}
    \newtheorem{solution}{解.}
}

{
    \theoremstyle{nonumberplain}
    \theoremheaderfont{\bfseries}
    \theorembodyfont{\normalfont}
    \theoremsymbol{\ensuremath{\blacksquare}}
    \newtheorem{proof}{证明.}
}

\crefname{exercise}{习题}{习题}
\crefname{figure}{图}{图}
\crefname{table}{表}{表}
\crefname{equation}{式}{式}

\newcommand{\dif}{\mathrm{d}}
\newcommand{\differ}{\backslash}
\newcommand{\ptl}{\partial}
\newcommand{\R}{\mathbb{R}}
\newcommand{\N}{\mathbb{N}}
\newcommand{\C}{\mathbb{C}}
\newcommand{\Z}{\mathbb{Z}}
\renewcommand{\phi}{\varphi}
\renewcommand{\epsilon}{\varepsilon}
\newcommand{\abs}[1]{\left\vert#1\right\vert}
\newcommand{\norm}[1]{\left\Vert#1\right\Vert}
\newcommand{\expect}{\mathbb{E}}
\newcommand{\var}{\mathrm{Var}}
\newcommand{\prob}{\mathbb{P}}
\newcommand{\Exp}{\mathrm{Exp}}
\newcommand{\poi}{\mathrm{Poi}}

\begin{document}
    \begin{center}
        \LARGE\bfseries
        数理统计作业
    \end{center}

    \begin{exercise}
        \label{ex:1}
        如果$X_1,\cdots,X_n$是来自$N(\mu,\sigma^2)$的随机样本,求样本标准差 
        \begin{equation*}
            s=\sqrt{\frac{\sum_{i=1}^n(X_i-\tilde{X})}{n-1}}
        \end{equation*}
        的期望和方差.
    \end{exercise}

    \begin{exercise}
        \label{ex:2}
        如果$X_1,\cdots,x_n$是来自均匀分布总体$U(0,1)$的随机样本,求 
        \begin{equation*}
            Y=\left(\prod_{i=1}^n X_i\right)^{-1/n}
        \end{equation*}
        的分布.
    \end{exercise}

    \begin{exercise}
        \label{ex:3}
        设$X_1,X_2$是来自$N(0,1)$的随机样本,
        \begin{enumerate}[(1)]
            \item 求$(X_2-X_1)/\sqrt{2}$的分布.
            \item 求$X_1^2+X_2^2$的分布.
            \item 求$(X_1+X_2)^2/(X_2-X_1)^2$的分布.
            \item 求$(X_2+X_1)/\sqrt{(X_1-X_2)^2}$的分布.
            \item 求$X_1^2/X_2^2$的分布.
        \end{enumerate}
    \end{exercise}

    \begin{exercise}
        \label{ex:4}
        设$X_1,\cdots,X_n$是来自总体$X$的随机样本,$\expect(X)=\mu$,且$\var(X)=0.25$,假设至少以95\%的概率保证$\abs{\tilde{X}-\mu}<0.01$,
        问样本量$n$至少应取多大?
    \end{exercise}

    \begin{exercise}
        \label{ex:5}
        如果$X$服从自由度为$m$和$n$的$F$分布,
        \begin{enumerate}[(1)]
            \item 求$W=\frac{mX/n}{1+mX/n}$的分布.
            \item 根据(1)的结果,计算$X$的期望和方差.
        \end{enumerate}
    \end{exercise}
\end{document}