\documentclass[a4paper,oneside,12pt]{ctexart}
\usepackage{enumerate,geometry,graphicx,bm,mathrsfs,xcolor,varwidth,framed,amsfonts,amssymb,indentfirst,fancyhdr}
\usepackage[colorlinks,linkcolor=red,anchorcolor=blue,citecolor=blue,urlcolor=blue]{hyperref}
\usepackage[thmmarks,hyperref]{ntheorem}
\usepackage{amsmath}

\setlength{\headheight}{15pt}
\allowdisplaybreaks[4]
\linespread{1.5}
\geometry{centering,left=2.54cm,right=2.54cm,top=3.18cm,bottom=3.18cm}
\pagestyle{fancy}
\fancyhead[L]{\kaishu 强基数学001}
\fancyhead[C]{\kaishu 张卓立}
\fancyhead[R]{\kaishu 2204110786}

{
    \theoremstyle{plain}
    \theoremheaderfont{\normalfont\bfseries}
    \theorembodyfont{\kaishu}
    \newtheorem{exercise}{习题}[section]
}

{
    \theoremstyle{nonumberplain}
    \theoremheaderfont{\bfseries}
    \theorembodyfont{\normalfont}
    \newtheorem{solution}{解.}
}

{
    \theoremstyle{nonumberplain}
    \theoremheaderfont{\bfseries}
    \theorembodyfont{\normalfont}
    \theoremsymbol{\ensuremath{\blacksquare}}
    \newtheorem{proof}{证明.}
}

\newcommand{\dif}{\mathrm{d}}
\newcommand{\differ}{\backslash}
\newcommand{\ptl}{\partial}
\newcommand{\R}{\mathbb{R}}
\newcommand{\N}{\mathbb{N}}
\newcommand{\C}{\mathbb{C}}
\newcommand{\Z}{\mathbb{Z}}
\renewcommand{\phi}{\varphi}
\renewcommand{\epsilon}{\varepsilon}
\newcommand{\abs}[1]{\left\vert#1\right\vert}
\newcommand{\norm}[1]{\left\Vert#1\right\Vert}



\begin{document}

    \begin{center}
        \LARGE\bfseries
        PDE作业
    \end{center}

    \section{第一章}
    
    \begin{exercise}\label{ex:1.1}
        利用Gauss-Green公式证明
        \begin{enumerate}
            \item 若$u,v\in C^1(\Omega)\cap C(\bar{\Omega})$,则\begin{equation*}
                \int_\Omega u_{x_i}v\dif x=-\int_\Omega uv_{x_i}\dif x+\int_\Omega uvn_i\dif s.
            \end{equation*}
            \item 若$u\in C^2(\Omega)\cap C^1(\bar{\Omega})$,则\begin{equation*}
                \int_\Omega \Delta u\dif x=\int_{\ptl\Omega}\frac{\ptl u}{\ptl n}\dif s.
            \end{equation*}
            \item 若$u,v\in C^2(\Omega)\cap C^1(\bar{\Omega})$,则\begin{equation*}
                \int_\Omega u\Delta v-v\Delta u\dif x=\int_{\ptl \Omega}u\frac{\ptl v}{\ptl n}-v\frac{\ptl u}{\ptl n}\dif s.
            \end{equation*}
        \end{enumerate}
    \end{exercise}

    \begin{exercise}
        \label{ex:1.2}
        将下列方程化为标准型.
        \begin{enumerate}
            \item $\sum_{i=1}^nu_{x_ix_i}+\sum_{1\leqslant j<j\leqslant n}u_{x_ix_j}=0$.
            \item $u_{xx}+2u_{xy}+2u_{yy}=0$.
        \end{enumerate}
    \end{exercise}

    \begin{exercise}
        \label{ex:1.3}
        
    \end{exercise}
\end{document}