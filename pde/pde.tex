\documentclass[a4paper,oneside,12pt]{ctexart}
\usepackage{enumerate,geometry,graphicx,bm,mathrsfs,xcolor,varwidth,framed,amsfonts,amssymb,indentfirst,fancyhdr}
\usepackage[colorlinks,linkcolor=red,anchorcolor=blue,citecolor=blue,urlcolor=blue]{hyperref}
\usepackage[thmmarks,hyperref]{ntheorem}
\usepackage{amsmath}
\usepackage{cleveref}

\setlength{\headheight}{15pt}
\allowdisplaybreaks[4]
\linespread{1.5}
\geometry{centering,left=2.54cm,right=2.54cm,top=3.18cm,bottom=3.18cm}
\pagestyle{fancy}
\fancyhead[L]{\kaishu 强基数学001}
\fancyhead[C]{\kaishu 张卓立}
\fancyhead[R]{\kaishu 2204110786}

{
    \theoremstyle{plain}
    \theoremheaderfont{\normalfont\bfseries}
    \theorembodyfont{\kaishu}
    \theoremseparator{.}
    \newtheorem{exercise}{习题}
}

{
    \theoremstyle{nonumberplain}
    \theoremheaderfont{\bfseries}
    \theorembodyfont{\normalfont}
    \newtheorem{solution}{解.}
}

{
    \theoremstyle{nonumberplain}
    \theoremheaderfont{\bfseries}
    \theorembodyfont{\normalfont}
    \theoremsymbol{\ensuremath{\blacksquare}}
    \newtheorem{proof}{证明.}
}

\crefname{exercise}{习题}{习题}
\crefname{figure}{图}{图}
\crefname{table}{表}{表}
\crefname{equation}{式}{式}

\newcommand{\dif}{\mathrm{d}}
\newcommand{\differ}{\backslash}
\newcommand{\ptl}{\partial}
\newcommand{\R}{\mathbb{R}}
\newcommand{\N}{\mathbb{N}}
\newcommand{\C}{\mathbb{C}}
\newcommand{\Z}{\mathbb{Z}}
\renewcommand{\phi}{\varphi}
\renewcommand{\epsilon}{\varepsilon}
\newcommand{\abs}[1]{\left\vert#1\right\vert}
\newcommand{\norm}[1]{\left\Vert#1\right\Vert}
\newcommand{\diag}{\mathrm{diag}}
\newcommand{\cond}[2]{\left. #1\right|_{#2}}



\begin{document}

    \begin{center}
        \LARGE\bfseries
        PDE作业
    \end{center}

    \subsection*{第二章第一次}

    \begin{exercise}
        \label{ex:3}
        用特征线求解下述Cauchy问题:

        (2)$\begin{cases}
            u_t+2u_x+u=xt, & t>0,-\infty<x<\infty,\\
            \cond{u}{t=0}=2-x,&-\infty<x<\infty.
        \end{cases}$
    \end{exercise}

    \begin{solution}
        设$x(t)$是解,那么 
        \begin{equation*}
            \frac{\dif u}{\dif t}=\frac{\ptl u}{\ptl x}\frac{\dif x}{\dif t}+\frac{\ptl u}{\ptl t}.
        \end{equation*}
        令$\frac{\dif u}{\dif t}=xt-u,\frac{\dif x}{\dif t}=2$,那么 
        \begin{equation*}
            x(t)=2t+c,
        \end{equation*}
        则$\cond{u}{t=0}=2-c$,且 
        \begin{equation*}
            \frac{\dif u}{\dif t}=(2t+c)t-u.
        \end{equation*}
        解,得 
        \begin{equation*}
            u(x(t),t)=t^2+(c-2)(t-1)+c_1e^t.
        \end{equation*}
        综合上面两式可得
        \begin{equation*}
            u(x,t)=t^2+(x-2t-2)(t-1).
        \end{equation*}
    \end{solution}

    \begin{exercise}
        \label{ex:7}
        试证明Cauchy问题 
        \begin{equation*}
            \begin{cases}
                u_{tt}-u_{xx}=6(x+t),&\quad -\infty<x<\infty,t>x,\\
                \cond{u}{t=x}=0,\cond{u_t}{t=x}=u_1(x),&-\infty<x<\infty
            \end{cases}
        \end{equation*}
        有解的充要条件是$u_1(x)-3x^2=\mathrm{const}$,如有解,解不唯一.试问:若把初值给定在直线$t=ax$上,为什么在$a=\pm 1$与$a\neq\pm 1$的情况,
        关于存在唯一性的结论不一样?
    \end{exercise}

    \begin{proof}
        ``$\Rightarrow$''.令$\tilde{t}=t-x,\tilde{u}(x,\tilde{t})=u(x,t)$,那么 
        \begin{gather*}
            u_x=\tilde{u}_x-\tilde{u}_{\tilde{t}},\\
            u_t=\tilde{u}_{\tilde{t}},\\
            u_{xx}=\tilde{u}_{xx}-2\tilde{u}_{x\tilde{t}}+\tilde{u}_{\tilde{t}\tilde{t}},\\
            u_{tt}=\tilde{u}_{\tilde{t}\tilde{t}}.
        \end{gather*}
        得到关于$\tilde{u}$的方程:
        \begin{equation*}
            \begin{cases}
                2\tilde{u}_{x\tilde{t}}-\tilde{u}_{xx}=6(2x+\tilde{t}),\\
                \cond{\tilde{u}}{\tilde{t}=0}=0,\\
                \cond{\tilde{u}_{\tilde{t}}}{\tilde{t}=0}=u_1(x).
            \end{cases}
        \end{equation*}
        那么$2\tilde{u}_{\tilde{t}}-\tilde{u}_x=6x^2+6\tilde{t}x+c$,$\tilde{t}=0$代入,又因为$\tilde{u}(x,0)=0$,那么$\cond{\tilde{u}_x}{\tilde{t}=0}=0$,则
        \begin{equation*}
           u_1(x)=3x^2+c/2. 
        \end{equation*}

        ``$\Leftarrow$''.
        根据上面的过程,考虑方程
        \begin{equation*}
            \begin{cases}
                2\tilde{u}_{\tilde{t}}-\tilde{u}_x=6x^2+6\tilde{t}x+c,\\
                \cond{\tilde{u}}{\tilde{t}=0}=0,\\
                \cond{\tilde{u}_{\tilde{t}}}{\tilde{t}=0}=3x^2+c/2.
            \end{cases}
        \end{equation*}
        利用特征线法求解,得到 
        \begin{equation*}
            x(\tilde{t})=-\frac{1}{2}\tilde{t}+c_1,
        \end{equation*}
        且
        \begin{align*}
            \tilde{u}(x,\tilde{t})&=2\left(-\frac{1}{2}\tilde{t}+c_1\right)^3-6c_1\left(-\frac{1}{2}t^2+c_1\right)^2+\frac{c}{2}\tilde{t}+4c_1^3\\
            &=2x^3-6\left(x+\frac{\tilde{t}}{2}\right)x^2+\frac{c\tilde{t}}{2}+4\left(x+\frac{t}{2}\right)^3\\
            &=2x^3-6\left(\frac{x+t}{2}\right)x^2+\frac{c(t-x)}{2}+4\left(\frac{t+x}{2}\right)^3.
        \end{align*}
        代入原方程验证确实是解.

        考虑方程 
        \begin{equation*}
            \begin{cases}
                u_{tt}-u_{xx}=0,&\quad -\infty<x<\infty,t>x,\\
                \cond{u}{t=x}=0,\cond{u_t}{t=x}=0,&-\infty<x<\infty
            \end{cases}
        \end{equation*}
        经过相同的变换得到 
        \begin{equation*}
            \begin{cases}
                2\tilde{u}_{\tilde{t}}-\tilde{u}_x=c,\\
                \cond{\tilde{u}}{\tilde{t}=0}=0,\\
                \cond{\tilde{u}_{\tilde{t}}}{\tilde{t}=0}=c/2.
            \end{cases}
        \end{equation*}
        显然有非零解$\tilde{u}(x,\tilde{t})=\frac{c\tilde{t}}{2}$,则原方程解不唯一.

        当$a$取值不同的时候,解的唯一性不同的原因可能是$t=ax$与特征线的交点不同.
    \end{proof}

    \begin{exercise}
        \label{ex:8}
        若$u=u(x,y,z,t)$是波动方程初值问题 
        \begin{equation*}
            \begin{cases}
                u_{tt}-a^2(u_{xx}+u_{yy}+u_{zz})=0,\\
                \cond{u}{t=0}=f(x)+g(y),\\
                \cond{u}{t=0}=\phi(y)+\psi(z)
            \end{cases}
        \end{equation*}
        的解,试求解的表达式.
    \end{exercise}

    \begin{solution}
        根据Kirchoff公式,得到 
        \begin{align*}
            u(x,y,z,t)&=\frac{\ptl}{\ptl t}\left[\frac{1}{4\pi a^2t}\iint_{S_{at}(x,y,z)}(f(u)+g(v))\dif u\dif v\dif w\right]\\
            &\quad +\frac{1}{4\pi a^2t}\iint_{S_{at}(x,y,z)}(\phi(y)+\psi(z))\dif u\dif v\dif w.
        \end{align*}
        进一步化简,得到 
        \begin{align*}
            u(x,y,z,t)&=\frac{1}{2\pi a}\frac{\ptl }{\ptl t}\left[\iint_{\Sigma_{at}(x,y)}\frac{f(u)+g(v)}{\sqrt{a^2t^2-(u-x)^2-(v-y)^2}}\dif u\dif v\right]\\
            &\quad +\frac{1}{2\pi a}\iint_{\Sigma_{at}(y,z)}\frac{\phi(y)+\psi(z)}{\sqrt{a^2t^2-(v-y)^2-(w-z)^2}}\dif v\dif w.
        \end{align*}
    \end{solution}

    \begin{exercise}
        \label{ex:10}
        试利用唯一性结果直接证明:当初值$\phi(x),\psi(x)$是偶函数,非齐次项$f(x,t)$是$x$的偶函数时非齐次波动方程初值问题的解$u(x,t)$关于$x$也是
        偶函数.
    \end{exercise}

    \begin{proof}
        因为$u(-x,t)$也是方程组的解,如果$u$不是$x$的偶函数,那么方程有两个相异的解,矛盾.即$u$是关于$x$的偶函数.
    \end{proof}

    \begin{exercise}
        \label{ex:12}
        证明半无界问题 
        \begin{equation*}
            \begin{cases}
                \frac{\ptl^2 u}{\ptl t^2}-a^2\frac{\ptl^2 u}{\ptl x^2}=f(x,t),&0<x<\infty,t>0,\\
                \cond{u}{t=0}=\phi(x),\cond{u_t}{t=0}=\psi(x),& 0\leqslant x<\infty,\\
                \cond{u}{x=0}=\mu(t),& t\geqslant 0
            \end{cases}
        \end{equation*}
        解的唯一性.
    \end{exercise}

    \begin{proof}
        要证明解的唯一性,只需证明
        \begin{equation*}
            \begin{cases}
                \frac{\ptl^2 u}{\ptl t^2}-a^2\frac{\ptl^2 u}{\ptl x^2}=f_1(x,t)=0,&0<x<\infty,t>0,\\
                \cond{u}{t=0}=\phi_1(t)=0,\cond{u_t}{t=0}=\psi_1(t)=0,& 0\leqslant x<\infty,\\
                \cond{u}{x=0}=\mu_1(t)=0,& t\geqslant 0
            \end{cases}
        \end{equation*}
        只有零解.

        考虑上面问题的奇延拓,那么它在区域$\{at-x_0\leqslant x\leqslant x_0-at,0\leqslant t\leqslant T\},x_0>0,0<T\leqslant\frac{x_0}{a}$上满足能量不等式,即
        \begin{gather*}
            \int_{\Omega_\tau}u^2(x,\tau)\dif x\leqslant M_1\left[\int_{\Omega_0}(\phi_1^2+\psi_1^2+a^2\phi_{1x}^2)\dif x+\iint_{K_\tau}f_1^2(x,t)\dif x\dif t\right],\\
            \iint_{K_\tau}u^2(x,t)\dif x\dif t\leqslant M\left[\int_{\Omega_0}(\phi_1^2+\psi_1^2+a^2\phi_{1x}^2)\dif x+\iint_{K\tau}f_1^2(x,t)\dif x\dif t\right].
        \end{gather*}
        那么$u$在区域$\{0\leqslant x\leqslant x_0-at,0\leqslant t\leqslant T\}$内恒为0,根据$x_0$的任意性,原方程只有零解.
    \end{proof}
    
\end{document}

\begin{exercise}
    \label{ex:19}
    求解三维波动方程的Cauchy问题:
    \begin{equation*}
        \begin{cases}
            u_{tt}=a^2(u_{xx}+u_{yy}+u_{zz}),\\
            \cond{u}{t=0}=0,\quad t\geqslant 0,\\
            \cond{u_t}{t=0}=x^3+y^2z.
        \end{cases}
    \end{equation*}
\end{exercise}

\begin{solution}
    根据Kirchoff公式,解为 
    \begin{equation*}
        u(x,t)=\frac{1}{4\pi a^2t}\int_{S_{at}(0)}\psi(x+y)\dif y,
    \end{equation*}
    进一步计算,可得 
    \begin{equation*}
        u(x,t)=\frac{at^2}{3}\left[x_1^3+x_2^3x_3+(3x_1+x_3)\frac{a^2t^2}{3}\right].
    \end{equation*}
\end{solution}

\begin{exercise}
    \label{ex:20}
    用降维法导出一维波动方程Cauchy问题的求解公式.
\end{exercise}

\begin{solution}
    考虑一维波动方程
    \begin{equation*}
        \begin{cases}
            \frac{\ptl^2 u}{\ptl t^2}-a^2\frac{\ptl^2 u}{\ptl x^2}=f(x,t),&\R\times (0,\infty),\\
            \cond{u}{t=0}=\phi(x),&x\in\R,\\
            \cond{u_t}{t=0}=\psi(x),&x\in\R.
        \end{cases}
    \end{equation*}
    令$x_1=x,\tilde{u}(x_1,x_2,t)=u(x_1,t),\tilde{\phi}(x_1,x_2)=\phi(x_1),\tilde{\psi}(x_1,x_2)=\psi(x_1)$,那么原方程Cauchy问题
    化为二维的波动方程的Cauchy问题,得到解为 
    \begin{align*}
        \tilde{u}(x_1,x_2,t)&=\frac{1}{2\pi a}\frac{\ptl}{\ptl t}\left[\iint_{\Sigma_{at}(x)}\frac{\phi(y_1)}{\sqrt{a^2t^2-(y_1-x_1)^2-(y_2-x_2)^2}}\dif y\right]\\
        &\quad+\frac{1}{2\pi a}\iint_{\Sigma_{at}(x)}\frac{\psi(y_1)}{\sqrt{a^2t^2-(y_1-x_1)^2-(y_2-x_2)^2}}\dif y\\
        &\quad+\frac{1}{2\pi a}\int_0^t\iint_{\Sigma_{a(t-\tau)}(x)}\frac{f(y_1,\tau)}{\sqrt{a^2(t-\tau)^2-(y_1-x_1)^2-(y_2-x_2)^2}}\dif y\dif\tau.
    \end{align*}
    又因为 
    \begin{align*}
        &\iint_{\Sigma_{at}(x)}\frac{\phi(y_1)}{\sqrt{a^2t^2-(y_1-x_1)^2-(y_2-x_2)^2}}\dif y\\
        =&\int_{x_1-at}^{x_1+at}\dif y_1\int_{-\sqrt{a^2t^2-(y_1-x_1)^2}}^{\sqrt{a^2t^2-(y_1-x_1)^2}}\frac{\phi(y_1)}{\sqrt{a^2t^2-(y_1-x_1)^2-y_2^2}}\dif y_2\\
        =&\int_{x_1-at}^{x_1+at}\dif y_1\int_{-1}^1\frac{\phi(y_1)}{\sqrt{1-y_2^2}}\dif y_2\\
        =&\pi\int_{x_1-at}^{x_1+at}\phi(y_1)\dif y_1,
    \end{align*}
    同理对剩下的积分也有类似的推导,
    \begin{align*}
        u(x_1,t)&=\frac{1}{2a}\frac{\ptl }{\ptl t}\left(\int_{x_1-at}^{x_1+at}\phi(y_1)\dif y_1\right)+\frac{1}{2a}\int_{x_1-at}^{x_1+at}\psi(y_1)\dif y_1\\
        &\quad +\frac{1}{2a}\int_0^t\left(\int_{x_1-a(t-\tau)}^{x_1+a(t-\tau)}f(y_1,\tau)\dif y_1\right)\dif\tau.
    \end{align*}
\end{solution}

\begin{exercise}
    \label{ex:21}
    求解二维波动方程的Cauchy问题:
    \begin{equation*}
        \begin{cases}
            u_{tt}=a^2(u_{xx}+u_{yy}),\\
            \cond{u}{t=0}=x^2(x+y).\\
            \cond{u_t}{t=0}=0.
        \end{cases}
    \end{equation*}
\end{exercise}

\begin{solution}
    根据Poisson公式,可以得到
\end{solution}