\documentclass[a4paper,oneside,12pt]{ctexart}
\usepackage{enumerate,geometry,graphicx,bm,mathrsfs,xcolor,varwidth,framed,amsfonts,amssymb,indentfirst,fancyhdr}
\usepackage[colorlinks,linkcolor=red,anchorcolor=blue,citecolor=blue,urlcolor=blue]{hyperref}
\usepackage[thmmarks,hyperref]{ntheorem}
\usepackage{amsmath}
\usepackage{cleveref}

\setlength{\headheight}{15pt}
\allowdisplaybreaks[4]
\linespread{1.5}
\geometry{centering,left=2.54cm,right=2.54cm,top=3.18cm,bottom=3.18cm}
\pagestyle{fancy}
\fancyhead[L]{\kaishu 强基数学001}
\fancyhead[C]{\kaishu 张卓立}
\fancyhead[R]{\kaishu 2204110786}

{
    \theoremstyle{plain}
    \theoremheaderfont{\normalfont\bfseries}
    \theorembodyfont{\kaishu}
    \newtheorem{exercise}{习题}
}

{
    \theoremstyle{nonumberplain}
    \theoremheaderfont{\bfseries}
    \theorembodyfont{\normalfont}
    \newtheorem{solution}{解.}
}

{
    \theoremstyle{nonumberplain}
    \theoremheaderfont{\bfseries}
    \theorembodyfont{\normalfont}
    \theoremsymbol{\ensuremath{\blacksquare}}
    \newtheorem{proof}{证明.}
}

\crefname{exercise}{习题}{习题}
\crefname{figure}{图}{图}
\crefname{table}{表}{表}
\crefname{equation}{式}{式}

\newcommand{\dif}{\mathrm{d}}
\newcommand{\differ}{\backslash}
\newcommand{\ptl}{\partial}
\newcommand{\R}{\mathbb{R}}
\newcommand{\N}{\mathbb{N}}
\newcommand{\C}{\mathbb{C}}
\newcommand{\Z}{\mathbb{Z}}
\renewcommand{\phi}{\varphi}
\renewcommand{\epsilon}{\varepsilon}
\newcommand{\abs}[1]{\left\vert#1\right\vert}
\newcommand{\norm}[1]{\left\Vert#1\right\Vert}
\newcommand{\diag}{\mathrm{diag}}



\begin{document}

    \begin{center}
        \LARGE\bfseries
        PDE作业
    \end{center}

    \section{第一章}
    
    \begin{exercise}\label{ex:1.1}
        利用Gauss-Green公式证明
        \begin{enumerate}
            \item 若$u,v\in C^1(\Omega)\cap C(\bar{\Omega})$,则\begin{equation*}
                \int_\Omega u_{x_i}v\dif x=-\int_\Omega uv_{x_i}\dif x+\int_{\ptl\Omega} uvn_i\dif s.
            \end{equation*}
            \item 若$u\in C^2(\Omega)\cap C^1(\bar{\Omega})$,则\begin{equation*}
                \int_\Omega \Delta u\dif x=\int_{\ptl\Omega}\frac{\ptl u}{\ptl n}\dif s.
            \end{equation*}
            \item 若$u,v\in C^2(\Omega)\cap C^1(\bar{\Omega})$,则\begin{equation*}
                \int_\Omega u\Delta v-v\Delta u\dif x=\int_{\ptl \Omega}u\frac{\ptl v}{\ptl n}-v\frac{\ptl u}{\ptl n}\dif s.
            \end{equation*}
        \end{enumerate}
    \end{exercise}

    \begin{proof}
        1. 令$F=(0,0,\cdots,uv,\cdots,0)^T$,根据Gauss-Green公式, 
        \begin{align*}
            \int_\Omega u_{x_i}v\dif x=\int_\Omega \nabla\cdot F\dif x=\int_{\ptl\Omega}F\cdot n\dif s=\int_{\ptl\Omega}uvn_i\dif s.
        \end{align*}

        2.注意到$\Delta u=\nabla\cdot \nabla u$,那么
        \begin{equation*}
            \int_\Omega \Delta u\dif x=\int_\Omega \nabla\cdot\nabla u\dif x=\int_{\ptl\Omega} \nabla u\cdot n\dif s=\int_{\ptl\Omega}\frac{\ptl u}{\ptl n}\dif s.
        \end{equation*}

        3.根据第1问的结论,
        \begin{align*}
            \int_\Omega u\Delta v-v\Delta u\dif x&=\int_\Omega \sum_{i=1}^n\left(uv_{x_ix_i}-vu_{x_ix_i}\right)\dif x\\
            &=-\int_\Omega\sum_{i=1}^n(u_{x_i}v_{x_i}-u_{x_i}v_{x_i})\dif x+\int_{\ptl\Omega}\sum_{i=1}^n(uv_{x_i}n_i-vu_{x_i}n_i)\dif s\\
            &=\int_{\ptl\Omega}(u\nabla v\cdot n-v\nabla u\cdot n)\dif s=\int_{\ptl \Omega}u\frac{\ptl v}{\ptl n}-v\frac{\ptl u}{\ptl n}\dif s.
        \end{align*}
        \ 
    \end{proof}

    \begin{exercise}
        \label{ex:1.2}
        将下列方程化为标准型.
        \begin{enumerate}
            \item $\sum_{i=1}^nu_{x_ix_i}+\sum_{1\leqslant j<j\leqslant n}u_{x_ix_j}=0$.
            \item $u_{xx}+2u_{xy}+2u_{yy}=0$.
        \end{enumerate}
    \end{exercise}

    \begin{solution}
        (1) 令矩阵 
        \begin{equation*}
            A=\begin{bmatrix}
                1 & 1/2 & \cdots & 1/2\\
                1/2 & 1 & \cdots & 1/2\\
                \vdots & \vdots & \ddots & \vdots\\
                1/2 & 1/2 & \cdots &1 \\
            \end{bmatrix},\quad P=\begin{bmatrix}
                -1/\sqrt{2} & \cdots & -1/\sqrt{2} & 1/\sqrt{n}\\
                1/\sqrt{2} & \cdots & 0 & 1/\sqrt{n}\\
                \vdots & \ddots & \vdots & \vdots \\
                0 &\cdots &1/\sqrt{2} & 1/\sqrt{n}
            \end{bmatrix},
        \end{equation*}
        那么$P^TAP=\diag\{1/2,\cdots,1/2,(n+1)/2\}$,令
        \begin{align*}
            &y_1=-\frac{1}{\sqrt{2}}x_1-\frac{1}{\sqrt{2}}x_2-\cdots-\frac{1}{\sqrt{2}}x_{n-1}+\frac{1}{\sqrt{n}}x_n,\\
            &y_2=\frac{1}{\sqrt{2}}+\frac{1}{\sqrt{n}}x_n,\\
            &\cdots\cdots\\
            &y_n=\frac{1}{\sqrt{2}}x_{n-1}+\frac{1}{\sqrt{n}}x_n,
        \end{align*}
        那么标准型为
        \begin{equation*}
            \frac{1}{2}u_{y_1y_1}+\cdots+\frac{1}{2}u_{y_{n-1}y_{n-1}}+\frac{n+1}{2}u_{y_ny_n}=0.
        \end{equation*}

        (2) 令
        \begin{equation*}
            A=\begin{bmatrix}
                1 & 1\\
                1 & 2
            \end{bmatrix},\quad P=\begin{bmatrix}
                \frac{-1+\sqrt{5}}{\sqrt{10-2\sqrt{5}}} & \frac{-\sqrt{5}-1}{\sqrt{10+2\sqrt{5}}} \\
                \frac{2}{\sqrt{10-2\sqrt{5}}} & \frac{2}{\sqrt{10+2\sqrt{5}}}
            \end{bmatrix},
        \end{equation*}
        那么$P^TAP=\diag\left\{\frac{3+\sqrt{5}}{2},\frac{3-\sqrt{5}}{2}\right\}$,令
        \begin{align*}
            s=\frac{-1+\sqrt{5}}{\sqrt{10-2\sqrt{5}}}x+\frac{-\sqrt{5}-1}{\sqrt{10+2\sqrt{5}}}y,\\
            t=\frac{2}{\sqrt{10-2\sqrt{5}}}x+\frac{2}{\sqrt{10+2\sqrt{5}}}y.
        \end{align*}
        得到标准型为
        \begin{equation*}
            \frac{3+\sqrt{5}}{2}u_{ss}+\frac{3-\sqrt{5}}{2}u_{tt}=0.
        \end{equation*}
    \end{solution}

    \begin{exercise}
        \label{ex:1.3}
        设
        \begin{equation*}
            J(v)=\frac{1}{2}\int_\Omega(\abs{\nabla v}^2+v^2)\dif x+\frac{1}{2}\int_{\ptl\Omega}\alpha(x)v^2\dif s-\int_\Omega fv\dif x-\int_{\ptl\Omega}gv\dif s,
        \end{equation*}
        其中$\alpha(x)\geqslant 0$.考虑以下三个问题:
        
        问题I(变分问题):求$u\in M=C^1(\bar{\Omega})$,使得
        \begin{equation*}
            J(u)=\min_{v\in M}J(v).
        \end{equation*}

        问题II:求$u\in M=C^1(\bar{\Omega})$,使得它对于任意的$v\in M$,都满足
        \begin{equation*}
            \int_\Omega(\nabla u\cdot\nabla v+u\cdot v-fv)\dif x+\int_{\ptl\Omega}(\alpha(x)uv-gv)\dif s=0.
        \end{equation*}
        
        问题III(第三边值问题):求$u\in C^2(\Omega)\cap C^1(\bar{\Omega})$,满足以下边值问题
        \begin{equation*}
            \begin{cases}
                -\Delta u+u=f,&x\in\Omega,\\
                \frac{\ptl u}{\ptl n}+\alpha(x)u=g,&x\in\ptl\Omega.
            \end{cases}
        \end{equation*}

        \begin{enumerate}
            \item 证明问题I与问题II等价.
            \item 当$u\in C^2(\Omega)\cap C^1(\bar{\Omega})$时,证明问题I、II、III等价.
        \end{enumerate}
    \end{exercise}

    \begin{proof}
        (1) ``$\Rightarrow$''.令$j(\epsilon)=J(u+\epsilon v)$,则$j(\epsilon)\geqslant j(0)$,那么$j'(0)=0$.又
        \begin{align*}
            j(\epsilon)&=\frac{1}{2}\int_\Omega (\abs{\nabla(u+\epsilon )v}^2+(u+\epsilon v)^2)\dif x+\frac{1}{2}\int_{\ptl \Omega}\alpha(x)(u+\epsilon v)^2\dif s\\
            &\quad -\int_\Omega f(x)(u+\epsilon v)\dif x-\int_{\ptl \Omega}g(x)(u+\epsilon v)\dif s\\
            &=\frac{1}{2}\int_\Omega\left[\left(\frac{\ptl u}{\ptl x}+\epsilon\frac{\ptl v}{\ptl x}\right)^2+\left(\frac{\ptl u}{\ptl y}+\epsilon\frac{\ptl v}{\ptl y}\right)^2+(u+\epsilon v)^2\right]\dif x\\
            &\quad +\frac{1}{2}\int_{\ptl \Omega}\alpha(x)(u+\epsilon v)^2\dif s-\int_\Omega f(x)(u+\epsilon v)\dif x-\int_{\ptl \Omega}g(x)(u+\epsilon v)\dif s.
        \end{align*}
        那么 
        \begin{align*}
            j'(\epsilon)&=\int_\Omega\left[\left(\frac{\ptl u}{\ptl x}+\epsilon\frac{\ptl v}{\ptl x}\right)\frac{\ptl v}{\ptl x}+\left(\frac{\ptl u}{\ptl y}+\epsilon\frac{\ptl v}{\ptl y}\right)\frac{\ptl v}{\ptl y}+(u+\epsilon v)v\right]\dif x\\
            &\quad +\int_{\ptl \Omega}\alpha(x)(u+\epsilon v)v\dif s-\int_\Omega f(x)v\dif x-\int_{\ptl \Omega}g(v)v\dif s.
        \end{align*}
        则
        \begin{align*}
            j'(0)&=\int_\Omega\left(\frac{\ptl u}{\ptl x}\frac{\ptl v}{\ptl x}+\frac{\ptl u}{\ptl y}\frac{\ptl v}{\ptl y}+uv\right)\dif x+\int_{\ptl \Omega}\alpha(x) uv\dif s-\int_\Omega fv\dif x-\int_{\ptl \Omega}gv\dif s\\
            &=0.
        \end{align*}

        ``$\Leftarrow$''.
        \begin{align*}
            J(v)-J(u)&=\frac{1}{2}\int_\Omega (\abs{\nabla v}^2+v^2-\abs{\nabla u}^2-u^2)\dif x+\frac{1}{2}\int_{\ptl \Omega}\alpha(x)(v^2-u^2)\dif s\\
            &\quad -\int_\Omega f(v-u)\dif x-\int_{\ptl \Omega}g(v-u)\dif s\\
            &=\frac{1}{2}\int_\Omega\left[\left(\frac{\ptl v}{\ptl x}\right)^2+\left(\frac{\ptl v}{\ptl y}\right)^2+v^2-\left(\frac{\ptl u}{\ptl x}\right)^2-\left(\frac{\ptl u}{\ptl x}\right)^2-u^2\right]\dif x\\
            &\quad +\frac{1}{2}\int_{\ptl \Omega}\alpha(x)(v^2-u^2)\dif s-\int_\Omega f(v-u)\dif x-\int_{\ptl \Omega}g(v-u)\dif s.
         \end{align*}
         又已知
         \begin{gather*}
            \int_\Omega\left[\left(\frac{\ptl u}{\ptl x}\right)^2+\left(\frac{\ptl u}{\ptl y}\right)^2+u^2-fu\right]\dif x+\int_{\ptl\Omega}(\alpha(x)u^2-gu)\dif s=0,\\
            \int_\Omega\left(\frac{\ptl u}{\ptl x}\frac{\ptl v}{\ptl x}+\frac{\ptl u}{\ptl y}\frac{\ptl v}{\ptl y}+uv-fv\right)\dif x+\int_{\ptl\Omega}(\alpha(x)uv-gv)\dif s=0,
         \end{gather*}
         联立以上两式,可得
         \begin{equation*}
            J(v)-J(u)=\frac{1}{2}\int_{\ptl\Omega}\alpha(x)(u-v)^2\dif s+\frac{1}{2}\int_\Omega(\abs{\nabla u-\nabla v}^2+(u-v)^2)\dif x\geqslant 0.
         \end{equation*}

        (2) II$\Rightarrow$III.根据\cref{ex:1.1}第一问,
        \begin{equation*}
            \int_\Omega (u\cdot v-fv-v\Delta u)\dif x\dif y+\int_{\ptl\Omega}(\alpha(x)uv+\frac{\ptl u}{\ptl n}v-gv)\dif s=0.
        \end{equation*}
        选取$v\in C_0^\infty(\Omega)$,那么 
        \begin{equation*}
            \int_\Omega(u\cdot v-fv-v\Delta u)\dif x\dif y=0.
        \end{equation*}
        又因为$u-f-\Delta u$连续,那么 
        \begin{equation*}
            u-\Delta u=f.
        \end{equation*}
        将上式代入,得到,
        \begin{equation*}
            \int_{\ptl\Omega}(\alpha(x)uv+\frac{\ptl u}{\ptl n}v-gv)\dif s=0,\quad v\in M.
        \end{equation*}
        那么 
        \begin{equation*}
            \alpha(x)u+\frac{\ptl u}{\ptl n}-g=0.
        \end{equation*}
        
        III$\Rightarrow$II.将两式代入即证.
    \end{proof}

    \begin{exercise}
        \label{ex:1.4}
        若$u$是Laplace方程$\Delta u=0$的解,如果$u(x)$只是向jing向径$r=\abs{x}$的函数,即$u(x)=\tilde{u}(r)$,试写出$\tilde{u}(r)$适合的常微分
        方程.
    \end{exercise}

    \begin{exercise}
        \label{ex:1.5}
        1. 证明在自变量代换
        \begin{equation*}
            \begin{cases}
                \xi=x-at,\\
                \eta=x+at
            \end{cases}
        \end{equation*}
        下,波动方程$u_{tt}-a^2u_{xx}=0$具有形式
        \begin{equation*}
            u_\tau=a^2u_{\xi\xi}.
        \end{equation*}
    \end{exercise}
\end{document}