\documentclass[a4paper,oneside,12pt]{ctexart}
\usepackage{enumerate,geometry,graphicx,bm,mathrsfs,xcolor,varwidth,framed,amsfonts,amssymb,indentfirst,fancyhdr}
\usepackage[colorlinks,linkcolor=red,anchorcolor=blue,citecolor=blue,urlcolor=blue]{hyperref}
\usepackage[thmmarks,hyperref]{ntheorem}
\usepackage{amsmath}
\usepackage{physics}
\usepackage{cleveref}

\setlength{\headheight}{15pt}
\allowdisplaybreaks[4]
\linespread{1.5}
\geometry{centering,left=2.54cm,right=2.54cm,top=3.18cm,bottom=3.18cm}
\pagestyle{fancy}
\fancyhead[L]{\kaishu 强基数学001}
\fancyhead[C]{\kaishu 张卓立}
\fancyhead[R]{\kaishu 2204110786}

{
    \theoremstyle{plain}
    \theoremheaderfont{\normalfont\bfseries}
    \theorembodyfont{\kaishu}
    \theoremseparator{.}
    \newtheorem{exercise}{习题}
}

{
    \theoremstyle{nonumberplain}
    \theoremheaderfont{\bfseries}
    \theorembodyfont{\normalfont}
    \newtheorem{solution}{解.}
}

{
    \theoremstyle{nonumberplain}
    \theoremheaderfont{\bfseries}
    \theorembodyfont{\normalfont}
    \theoremsymbol{\ensuremath{\blacksquare}}
    \newtheorem{proof}{证明.}
}

\crefname{exercise}{习题}{习题}
\crefname{figure}{图}{图}
\crefname{table}{表}{表}
\crefname{equation}{式}{式}

\newcommand{\differ}{\backslash}
\newcommand{\ptl}{\partial}
\newcommand{\R}{\mathbb{R}}
\newcommand{\N}{\mathbb{N}}
\newcommand{\C}{\mathbb{C}}
\newcommand{\Z}{\mathbb{Z}}
\renewcommand{\phi}{\varphi}
\renewcommand{\epsilon}{\varepsilon}
\newcommand{\diag}{\mathrm{diag}}



\begin{document}

    \begin{center}
        \LARGE\bfseries
        PDE作业
    \end{center}

    \subsection*{第二章第二次}

    \begin{exercise}
        \label{ex:19}
        求解三维波动方程的Cauchy问题:
        \begin{equation*}
            \begin{cases}
                u_{tt}=a^2(u_{xx}+u_{yy}+u_{zz}),\\
                \eval{u}_{t=0}=0,\quad t\geqslant 0,\\
                \eval{u_t}_{t=0}=x^3+y^2z.
            \end{cases}
        \end{equation*}
    \end{exercise}
    
    \begin{solution}
        根据Kirchoff公式,解为 
        \begin{equation*}
            u(x,t)=\frac{1}{4\pi a^2t}\int_{S_{at}(0)}\psi(x+y)\dd y,
        \end{equation*}
        进一步计算,可得 
        \begin{equation*}
            u(x,t)=\frac{at^2}{3}\left[x_1^3+x_2^3x_3+(3x_1+x_3)\frac{a^2t^2}{3}\right].
        \end{equation*}
    \end{solution}
    
    \begin{exercise}
        \label{ex:20}
        用降维法导出一维波动方程Cauchy问题的求解公式.
    \end{exercise}
    
    \begin{solution}
        考虑一维波动方程
        \begin{equation*}
            \begin{cases}
                \frac{\ptl^2 u}{\ptl t^2}-a^2\frac{\ptl^2 u}{\ptl x^2}=f(x,t),&\R\times (0,\infty),\\
                \eval{u}_{t=0}=\phi(x),&x\in\R,\\
                \eval{u_t}_{t=0}=\psi(x),&x\in\R.
            \end{cases}
        \end{equation*}
        令$x_1=x,\tilde{u}(x_1,x_2,t)=u(x_1,t),\tilde{\phi}(x_1,x_2)=\phi(x_1),\tilde{\psi}(x_1,x_2)=\psi(x_1)$,那么原方程Cauchy问题
        化为二维的波动方程的Cauchy问题,得到解为 
        \begin{align*}
            \tilde{u}(x_1,x_2,t)&=\frac{1}{2\pi a}\frac{\ptl}{\ptl t}\left[\iint_{\Sigma_{at}(x)}\frac{\phi(y_1)}{\sqrt{a^2t^2-(y_1-x_1)^2-(y_2-x_2)^2}}\dd y\right]\\
            &\quad+\frac{1}{2\pi a}\iint_{\Sigma_{at}(x)}\frac{\psi(y_1)}{\sqrt{a^2t^2-(y_1-x_1)^2-(y_2-x_2)^2}}\dd y\\
            &\quad+\frac{1}{2\pi a}\int_0^t\iint_{\Sigma_{a(t-\tau)}(x)}\frac{f(y_1,\tau)}{\sqrt{a^2(t-\tau)^2-(y_1-x_1)^2-(y_2-x_2)^2}}\dd y\dd\tau.
        \end{align*}
        又因为 
        \begin{align*}
            &\iint_{\Sigma_{at}(x)}\frac{\phi(y_1)}{\sqrt{a^2t^2-(y_1-x_1)^2-(y_2-x_2)^2}}\dd y\\
            =&\int_{x_1-at}^{x_1+at}\dd y_1\int_{-\sqrt{a^2t^2-(y_1-x_1)^2}}^{\sqrt{a^2t^2-(y_1-x_1)^2}}\frac{\phi(y_1)}{\sqrt{a^2t^2-(y_1-x_1)^2-y_2^2}}\dd y_2\\
            =&\int_{x_1-at}^{x_1+at}\dd y_1\int_{-1}^1\frac{\phi(y_1)}{\sqrt{1-y_2^2}}\dd y_2\\
            =&\pi\int_{x_1-at}^{x_1+at}\phi(y_1)\dd y_1,
        \end{align*}
        同理对剩下的积分也有类似的推导,
        \begin{align*}
            u(x_1,t)&=\frac{1}{2a}\frac{\ptl }{\ptl t}\left(\int_{x_1-at}^{x_1+at}\phi(y_1)\dd y_1\right)+\frac{1}{2a}\int_{x_1-at}^{x_1+at}\psi(y_1)\dd y_1\\
            &\quad +\frac{1}{2a}\int_0^t\left(\int_{x_1-a(t-\tau)}^{x_1+a(t-\tau)}f(y_1,\tau)\dd y_1\right)\dd\tau.
        \end{align*}
    \end{solution}
    
    \begin{exercise}
        \label{ex:21}
        求解二维波动方程的Cauchy问题:
        \begin{equation*}
            \begin{cases}
                u_{tt}=a^2(u_{xx}+u_{yy}),\\
                \eval{u}_{t=0}=x^2(x+y).\\
                \eval{u_t}_{t=0}=0.
            \end{cases}
        \end{equation*}
    \end{exercise}
    
    \begin{solution}
        根据Poisson公式,可以得到 
        \begin{equation*}
            u(x,y,t)=\frac{1}{2\pi a}\pdv{t}\left(\iint_{\Sigma_{at}(x,y)}\frac{x_1^2(x_1+y_1)}{\sqrt{a^2t^2-(x_1-x)^2-(y_1-y)^2}}\dd{y_1}\dd{y_2}\right),
        \end{equation*}
        进一步计算, 
        \begin{equation*}
            u(x,y,t)=a^2t^2(x+y)+x^2(x+y)+\frac{3\pi a^2t^2x}{4}.
        \end{equation*}
    \end{solution}

    \begin{exercise}
        \label{ex:22.(4)}
        求一下特征值问题的特征函数:

        (4) $\begin{cases}
            X''(x)+\lambda X(x)=0, & 0<x<l,\\
            X(0)=X'(l)+hX(l)=0\ (h>0\text{是常数}).
        \end{cases}$
    \end{exercise}

    \begin{solution}
        因为特征值问题的特征值均大于0,那么 
        \begin{equation*}
            X(x)=c_1\sin(\sqrt{\lambda}x)+c_2\cos(\sqrt{\lambda} x),
        \end{equation*}
        代入条件得$c_2=0,\sqrt{\lambda}\cos(\sqrt{\lambda}l)+h\sin(\sqrt{\lambda}l)=0$,那么$\sqrt{\lambda}l\neq\frac{\pi}{2}+k\pi$,即$\lambda$是
        $\sqrt{\lambda}=-h\tan(\sqrt{\lambda}l)$的正数解,且方程有正数解$\lambda_n$,那么特征方程为 
        \begin{equation*}
            X_n(x)=\sin(\sqrt{\lambda_n}x).
        \end{equation*}
    \end{solution}

    \begin{exercise}
        \label{ex:23.(2)}
        用分离变量法求解下列定解问题:

        (2) $\begin{cases}
            Lu=0, & (x,t)\in Q,\\
            \eval{u}_{x=0}=\eval{u_x}_{x=l}=0, & t\geqslant 0,\\
            \eval{u}_{t=0}=x(x-2l),\eval{u_t}_{t=0}=0, & 0\leqslant x\leqslant l.
        \end{cases}$
    \end{exercise}

    \begin{solution}
        令$v=u-x(x-2l)$, 
        \begin{equation*}
            \begin{cases}
                Lv=2a^2,\\
                \eval{v}_{x=0}=\eval{v_x}_{x=0}=0,\\
                \eval{v}_{t=0}=\eval{v_t}_{t=0}=0.
            \end{cases}
        \end{equation*}
        令$v(x,t)=X(x)T(t)$,那么 
        \begin{equation*}
            \frac{T''(t)}{a^2T(t)}=\frac{X''(x)}{X(x)}=-\lambda,
        \end{equation*}
        即 
        \begin{gather*}
            X''(x)+\lambda X(x)=0,\\
            T''(t)+\lambda a^2 T(t)=0.
        \end{gather*}
        且$X(0)=X'(l)=T(0)=T'(0)=0$.

        类似\cref{ex:22.(4)}可得$X_k(x)=\sin\frac{(2k+1)\pi}{2l}x$,令$v(x,t)=\sum_{k=1}^\infty T_k(t)\sin\frac{(2k+1)\pi}{2l}x$和
        $2a^2=\sum_{k=1}^\infty f_k\sin\frac{(2k+1)\pi}{2l}x=\sum_{k=1}^\infty \frac{8a^2}{(2k+1)\pi}\sin\frac{(2k+1)\pi}{2l}x$,则 
        \begin{equation*}
            \begin{cases}
                T''_k(t)+a^2T_k(t)\left(\frac{(2k+1)\pi}{2l}\right)^2=\frac{8a^2}{(2k+1)\pi},\\
                T_k(0)=0,T_k'(0)=0.
            \end{cases}
        \end{equation*}
        解,得$T_k(t)=\frac{32l^2}{((2k+1)\pi)^3}\left(1-\cos a\left(\frac{(2k+1)\pi}{2l}\right)t\right)$,即 
        \begin{equation*}
            v(x,t)=\sum_{k=1}^\infty \frac{32l^2}{((2k+1)\pi)^3}\left(1-\cos a\left(\frac{(2k+1)\pi}{2l}\right)t\right)\sin\frac{(2k+1)\pi}{2l}x,
        \end{equation*}
        那么
        \begin{equation*}
            u(x,t)=x(x-2l)+\sum_{k=1}^\infty \frac{32l^2}{((2k+1)\pi)^3}\left(1-\cos a\left(\frac{(2k+1)\pi}{2l}\right)t\right)\sin\frac{(2k+1)\pi}{2l}x.
        \end{equation*}
    \end{solution}

    \begin{exercise}
        \label{ex:25}
        设$u(x,t)$适合定解问题: 
        \begin{equation*}
            \begin{cases}
                Lu=f(x,t), & (x,t)\in Q,\\
                -\pdv{u}{x}+\alpha\eval{u}_{x=0}=\mu_1(t), & t\geqslant 0,\\
                \pdv{u}{x}+\beta\eval{u}_{x=l}=\mu(t), & t\geqslant 0,\\
                \eval{u}_{t=0}=\phi(x),\eval{u_t}_{t=0}=\psi(x), & 0\leqslant x\leqslant l,
            \end{cases}
        \end{equation*}
        试引进辅助函数,把边值条件齐次化,设 
        
        (a) $a>0,\beta>0$; \quad (b) $\alpha=\beta=0$.
    \end{exercise}

    \begin{solution}
        (a) 令
        \begin{equation}
            \label{eq:25a齐次化}
            v(x,t)=\frac{l-x}{l}(-u_x(x,t)+\alpha u_1(x,t)-\mu_1(t))+\frac{x}{l}(u_x(x,t)+\beta u(x,t)-\mu(t)),
        \end{equation}
        则$\eval{v}_{x=0}=-u_x(0,t)+\alpha u(0,t)-\mu_1(t)=0,\eval{v}_{x=l}=u_x(l,t)+\beta u(x,t)-\mu(t)=0$.

        (b) 令\cref{eq:25a齐次化}中的$\alpha=\beta=0$,$v(x,t)=\frac{l-x}{l}(-u_x(x,t)-\mu_1(t))+\frac{x}{l}(u_x(x,t)-\mu(t))$.
    \end{solution}

    \begin{exercise}
        \label{ex:26.(2)}
        用分离变量法求解下列定解问题: 
        
        (2) $\begin{cases}
            Lu=-2b\pdv{u}{t}+g, & (x,t)\in Q,\\
            \eval{u}_{x=0}=\eval{u}_{x=t}=0, & t\geqslant 0,\\
            \eval{u}_{t=0}=\eval{u_t}_{t=0}=0, & 0\leqslant x\leqslant l.
        \end{cases}$
    \end{exercise}

    \begin{solution}
        首先先令$Lu=0$,然后令$U(x,t)=T(t)X(x)$,类似\cref{ex:22.(4)},得到$X_k(x)=\sin\frac{k\pi}{l}x,\\\lambda_n=\frac{n\pi }{l}$.

        对于原方程,令$u(x)=\sum_{n=1}^\infty T_n(t)\sin\frac{n\pi}{l}x,g=\sum_{n=1}^\infty g_n\sin\frac{k\pi}{l}x$,则 
        \begin{equation*}
            T_n''(t)+a^2\left(\frac{k\pi}{l}\right)^2T_n(t)+2bT'_n(t)=g_n,
        \end{equation*}
        且$T_n(0)=0,T_n'(0)=0$,又因为$g_n=\frac{1-(-1)^n}{n\pi g}$,求解上面的常微分方程,得到:

        1. 当$\Delta=4b^2-4a^2\left(\frac{n\pi}{l}\right)^2\geqslant 0$时, 
        \begin{align*}
            T_n(t)&=\frac{l^2g_n}{a^2n^2\pi^2}+\frac{(b-\sqrt{b^2-a^2\lambda_n^2})g_n}{2\lambda_n^2a^2\sqrt{b^2-a^2\lambda_n^2}}\exp(\left(-b-\sqrt{b^2-a^2\lambda_n^2}\right)t)\\
            &\quad +\frac{(b+\sqrt{b^2-a^2\lambda_n^2})g_n}{\lambda_n^2a^2}\exp((-b+\sqrt{b^2-a^2\lambda_n^2})t).
        \end{align*}

        2. 当$\Delta<0$时, 
        \begin{align*}
            T_n(t)&=\frac{l^2g_n}{a^2n^2\pi^2}-\frac{g_n}{\lambda_n^2a^2}e^{-bt}\cos(\sqrt{a^2\lambda_n^2-b^2}t)\\
            &\quad+\frac{bg_n}{\lambda_n^2a^2\sqrt{a^2\lambda_n^2-b^2}}e^{-bt}\sin(\sqrt{a^2\lambda_n^2-b^2}t).
        \end{align*}
        
        那么 
        \begin{align*}
            u(x,t)&=\sum_{n\leqslant\frac{bl}{a\pi}}\left[\frac{l^2g_n}{a^2n^2\pi^2}+\frac{(b-\sqrt{b^2-a^2\lambda_n^2})g_n}{2\lambda_n^2a^2\sqrt{b^2-a^2\lambda_n^2}}\exp(\left(-b-\sqrt{b^2-a^2\lambda_n^2}\right)t)\right.\\
            &\quad +\left.\frac{(b+\sqrt{b^2-a^2\lambda_n^2})g_n}{\lambda_n^2a^2}\exp((-b+\sqrt{b^2-a^2\lambda_n^2})t)\right]\sin\frac{n\pi}{l}x\\
            &\quad +\sum_{n>\frac{bl}{a\pi}}\left[\frac{l^2g_n}{a^2n^2\pi^2}-\frac{g_n}{\lambda_n^2a^2}e^{-bt}\cos(\sqrt{a^2\lambda_n^2-b^2}t)\right.\\
            &\quad +\left.\frac{bg_n}{\lambda_n^2a^2\sqrt{a^2\lambda_n^2-b^2}}e^{-bt}\sin(\sqrt{a^2\lambda_n^2-b^2}t)\right]\sin\frac{n\pi}{l}x.
        \end{align*}
    \end{solution}

    \begin{exercise}
        \label{ex:27}
        考虑定解问题: 
        \begin{equation*}
            \begin{cases}
                u_{tt}-u_{xx}=f(x,t), & (x,t)\in Q,\\
                \eval{u}_{x=0}=\eval{u}_{x=l}=0, & 0\leqslant t\leqslant T,\\
                \eval{u}_{t=0}=\phi(x),\eval{u_t}_{t=0}\psi(x), & 0\leqslant x\leqslant l.
            \end{cases}
        \end{equation*}
        试问对$\phi,\psi,f$加什么条件才能保证由Fourier方法所得的解是古典解?试证明之!
    \end{exercise}

    \begin{proof}
        根据条件,$\phi$至少二阶连续可微,$\psi$至少连续可微,因为$u\in C^2$,那么$\lim_{x\to 0}u(x,0)\\=\lim_{t\to 0}u(0,t)$,则$\phi(0)=0$,同样可得$\phi(l)=0$.又因为$u_t(0,t)=u_t(l,t)=0$,那么类似
        $\psi(0)=\psi(l)=0$.又因为$u_{xx}(x,0)=\phi''(x),u_{tt}(0,t)=u_{tt}(l,t)=0$,则$\phi''(0)=-f(0,0),\phi''(l)=-f(l,0)$.

        可将原问题分为三部分.

        (1) $\begin{cases}
            Lu=0\\
            \eval{u}_{t=0}=\phi(x),\eval{u_t}_{t=0}=0\\
            \eval{u}_{x=0}=0,\eval{u}_{x=l}=0
        \end{cases}$,解是$u_1$.

        (2) $\begin{cases}
            Lu=0\\
            \eval{u}_{t=0}=0,\eval{u_t}_{t=0}=\psi(x)\\
            \eval{u}_{x=0}=0,\eval{u}_{x=l}=0
        \end{cases}$,解是$u_2$.

        (3) $\begin{cases}
            Lu=f(x,t)\\
            \eval{u}_{t=0}=0,\eval{u_t}_{t=0}=0\\
            \eval{u}_{x=0}=0,\eval{u}_{x=l}=0
        \end{cases}$,解是$u_3$.

        原方程解$u=u_1+u_2+u_3$.且$u_1,u_2,u_3$满足$u_2=M_\phi(x,t),u_1=\pdv{t}M_\phi(x,t),u_3=\int_0^tM_{f_\tau}(t,x;\tau)\dd{\tau}$.
        当$\phi\in C^3,\psi\in C^2$时,$u_1,u_2\in C^2$,要让$u_3\in C^2$,只需使$M_{f_\tau}(t,x;\tau)\in C^2$,它是 
        \begin{equation*}
            \begin{cases}
                Lu=0\\
            \eval{u}_{t=\tau}=\phi(x),\eval{u_t}_{t=\tau}=f(x,\tau)\\
            \eval{u}_{x=0}=0,\eval{u}_{x=l}=0
            \end{cases}
        \end{equation*}
        的解,又因为$0\leqslant \tau\leqslant t$,则$f\in C^2,f(0,t)=f(l,t)=0$.
    \end{proof}
    
    \begin{exercise}
        \label{ex:28}
        用能量不等式证明一维波动方程带有第三边值条件的初边值问题解的唯一性.
    \end{exercise}

    \begin{proof}
        考虑方程 
        \begin{equation*}
            \begin{cases}
                Lu=f(x,t)\\
                \eval{u}_{t=0}=\phi(x)\\
                \eval{u_t}_{t=0}=\psi(x)\\
                -\eval{\pdv{u}{x}}_{x=0}+\alpha_1 u(0,t)=g_1(t)\\
                \eval{\pdv{u}{x}}_{x=l}+\alpha_2 u(l,t)=g_2(t)
            \end{cases}
        \end{equation*}
        因为这个方程的解集是方程
        \begin{equation}
            \label{eq:28.转化后}
            \begin{cases}
                Lu=f(x,t)\\
                \eval{u}_{t=0}=\phi(x)\\
                \eval{u_t}_{t=0}=\psi(x)\\
            \end{cases}
        \end{equation}
        的子集,且对这个方程
        \begin{equation*}
            \iint_{k_\tau}u^2(x,t)\dd{x}\dd{t}\leqslant M_1\left[\int_{\Omega_0}(\phi^2+\psi^2+a^2\phi_x^2)\dd{x}+\iint_{K_\tau}f^2\dd{x}\dd{t}\right],
        \end{equation*}
        则方程(\ref{eq:28.转化后})的解具有唯一性,即原方程解有唯一性.
    \end{proof}
\end{document}










\begin{exercise}
    \label{ex:13}
    证明以下Cauchy问题 
    \begin{equation*}
        \begin{cases}
            \pdv[2]{u}{t}-a^2\pdv[2]{u}{x}+b(x,t)\pdv{u}{x}+c(x,t)\pdv{u}{t}=f(x,t), & -\infty<x<\infty,t>0,\\
            \eval{u}_{t=0}=\phi(x),\eval{u_t}_{t=0}=\psi(x),-\infty<x<\infty
        \end{cases}
    \end{equation*}
    解的唯一性,其中$b(x,t),c(x,t)$都是有界连续函数.
\end{exercise}

\begin{proof}
    类似于能量不等式的证明过程,方程两边同时乘上$\pdv{u}{t}$,再在$K_\tau$上积分,得到 
    \begin{equation*}
        \iint_{K_\tau}\left[\frac{1}{2}\pdv{u}{t}\pdv[2]{u}{t}-a^2\pdv{u}{t}\pdv[2]{u}{x}+b(x,t)\pdv{u}{t}\pdv{u}{x}+c(x,t)\left(\pdv{u}{t}\right)^2\right]\dd{x}\dd{t}=0.
    \end{equation*}
    因为 
    \begin{gather*}
        \pdv{u}{t}\pdv[2]{u}{t}=\frac{1}{2}\pdv{t}\left(\pdv{u}{t}\right)^2,\\
        \pdv{u}{t}\pdv[2]{u}{x}=\pdv{x}\left(\pdv{u}{t}\pdv{u}{x}\right),
    \end{gather*}
    将上式代入原先的积分,
    \begin{align*}
        &\iint_{K_\tau}\left(\frac{1}{2}\pdv{t}\left(\pdv{u}{t}\right)^2-a^2\pdv{x}\left(\pdv{u}{t}\pdv{u}{x}\right)+\frac{a^2}{2}\pdv{t}\left(\pdv{u}{x}\right)^2\right)\dd{x}\dd{t}\\
        &\quad +\iint_{K_\tau}\left[c(x,t)\left(\pdv{u}{t}\right)^2+b(x,t)\pdv{u}{t}\pdv{u}{x}\right]\dd{x}\dd{t}=0.
    \end{align*}
    用Green公式,类似可以得到存在$M\geqslant 0$,
    \begin{equation*}
        \int_{\Omega_\tau}\left[\left(\pdv{u}{t}\right)^2+a^2\left(\pdv{u}{x}\right)^2\right]\dd{x}\leqslant 2M\iint_{K\tau}\left[\left(\pdv{u}{t}\right)^2+\pdv{u}{t}\pdv{u}{x}\right]\dd{x}\dd{t},
    \end{equation*}
    对右式$\pdv{u}{t}\pdv{u}{x}$用均值不等式,那么存在$k$,使得
    \begin{equation*}
        \int_{\Omega_\tau}\left[\left(\pdv{u}{t}\right)^2+k^2\left(\pdv{u}{x}\right)^2\right]\dd{x}\leqslant 2M'\iint_{K\tau}\left[\left(\pdv{u}{t}\right)^2+k^2\left(\pdv{u}{x}\right)^2\right]\dd{x}\dd{t}
    \end{equation*}
    应用Gronwall不等式可以得到$u_t=u_x=0$,又因为
    \begin{equation*}
        \iint_{K_\tau}u_tu\dd{x}\dd{t}=\iint_{K_\tau}\pdv{t}\left(\frac{u^2}{2}\right)\dd{x}\dd{t}=\frac{1}{2}\int_{\Omega_\tau}u^2(x,\tau)\dd{x},
    \end{equation*}
    再次用均值不等式和Gronwall不等式可以得到$u\equiv 0$.
\end{proof}
