\documentclass[a4paper,oneside,12pt]{ctexart}
\usepackage{float,enumerate,setspace,geometry,graphicx,bm,mathrsfs,xcolor,varwidth,framed,amsfonts,amssymb,indentfirst,fancyhdr,BOONDOX-cal}
\usepackage[colorlinks,linkcolor=red,anchorcolor=blue,citecolor=blue,urlcolor=blue]{hyperref}
\usepackage[thmmarks,hyperref]{ntheorem}
\usepackage{amsmath}
\usepackage{mathtools}
\usepackage{cleveref}
\usepackage{physics}
\usepackage{subcaption}
\usepackage{tikz,pgfplots}
\usepackage{asymptote}

\pgfplotsset{compat=1.15}
\usetikzlibrary{arrows}
\setlength{\headheight}{15pt}
\allowdisplaybreaks[4]
\onehalfspacing
\geometry{centering,left=2.54cm,right=2.54cm,top=3.18cm,bottom=3.18cm}
\pagestyle{fancy}
\fancyhead[L]{\kaishu 强基数学001}
\fancyhead[C]{\kaishu 张卓立}
\fancyhead[R]{\kaishu 2204110786}

\crefname{exercise}{习题}{习题}
\crefname{figure}{图}{图}
\crefname{table}{表}{表}
\crefname{equation}{式}{式}
\crefdefaultlabelformat{(#2#1#3)}

{
    \theoremstyle{plain}
    \theoremheaderfont{\normalfont\bfseries}
    \theorembodyfont{\kaishu}
    \theoremseparator{.}
    \newtheorem{exercise}{习题}
}

{
    \theoremstyle{nonumberplain}
    \theoremheaderfont{\bfseries}
    \theorembodyfont{\normalfont}
    \theoremseparator{.}
    \newtheorem{solution}{解}
}

{
    \theoremstyle{nonumberplain}
    \theoremheaderfont{\bfseries}
    \theorembodyfont{\normalfont}
    \theoremsymbol{\ensuremath{\blacksquare}}
    \theoremseparator{.}
    \newtheorem{proof}{证明}
}

\newcommand{\differ}{\backslash}
\newcommand{\ptl}{\partial}
\newcommand{\R}{\mathbb{R}}
\newcommand{\N}{\mathbb{N}}
\newcommand{\C}{\mathbb{C}}
\newcommand{\Z}{\mathbb{Z}}
\newcommand{\sX}{\mathscr{X}}
\newcommand{\sY}{\mathscr{Y}}
\renewcommand{\P}{\mathbb{P}}
\newcommand{\K}{\mathbb{K}}
\renewcommand{\i}{\mathrm{i}}
\renewcommand{\phi}{\varphi}
\renewcommand{\epsilon}{\varepsilon}
\renewcommand{\emptyset}{\varnothing}
\renewcommand{\liminf}{\varliminf}
\renewcommand{\limsup}{\varlimsup}
\renewcommand{\ae}{\ \mathrm{a.e.}\ }
\newcommand{\diam}{\mathrm{diam\ }}
\newcommand{\toinmeasure}[1]{\underset{#1}{\Rightarrow}}
\newcommand{\DEF}{\overset{\triangle}{=}}
\newcommand{\sL}{\mathscr{L}}
\newcommand{\leftperp}{\prescript{\perp}{}}

\DeclareMathOperator{\Span}{span}
\DeclareMathOperator{\Ker}{Ker}

\begin{document}
    
    \begin{center}
        \bfseries\LARGE
        泛函分析作业
    \end{center}

    \begin{exercise}
        \label{ex:1 on note}
        当$\sX$是复Hilbert空间, $T\in\sL(\sX)$, $T^\ast=T\Leftrightarrow (Tx,x)\in\R,\forall x\in\sX$.
    \end{exercise}

    \begin{exercise}
        \label{ex:2 on note}
        设$\sX$是Hilbert空间, $T_1,T_2\in\sL(\sX)$, $T_1^\ast=T_1,T_2^\ast=T_2$, $T_1T_2=T_2T_1\Leftrightarrow (T_1T_2)^\ast=T_1T_2$.
    \end{exercise}

    \begin{exercise}
        \label{ex:3 on note}
        设$\sX$是Hilbert空间, $T\in\sL(\sX)$, 证明: $\Ker(T^\ast)=R(T)^\perp=\left(\overline{R(T)}\right)^\perp$.
    \end{exercise}

    \begin{exercise}
        \label{ex:4 on note}
        证明: $^\perp(M)^\perp=\overline{M}$.
    \end{exercise}

    \begin{exercise}
        \label{ex:5 on note}
        设$\sX,\sY$是$B^\ast$空间, $T\in\sL(\sX,\sY)$, 则$\Ker(T^\ast)=\leftperp R(T),\Ker(T)=R(T^\ast)^\perp$.
    \end{exercise}

    \begin{exercise}
        \label{ex:6 on note}
        设$X=\left\{\xi=(x_1,x_2,\cdots)\in l^2:\sum_{n=1}^\infty\abs{nx_n}^2<\infty\right\},\norm{\xi}_X=\left(\sum_{n=1}^\infty \abs{nx_n}^2\right)^{\frac{1}{2}}$.
    \end{exercise}

\end{document}