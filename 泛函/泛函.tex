\documentclass[a4paper,oneside,12pt]{ctexart}
\usepackage{enumerate,geometry,graphicx,bm,mathrsfs,xcolor,varwidth,framed,amsfonts,amssymb,indentfirst,fancyhdr,BOONDOX-cal}
\usepackage[colorlinks,linkcolor=red,anchorcolor=blue,citecolor=blue,urlcolor=blue]{hyperref}
\usepackage[thmmarks,hyperref]{ntheorem}
\usepackage{amsmath}

\setlength{\headheight}{15pt}
\allowdisplaybreaks[4]
\linespread{1.5}
\geometry{centering,left=2.54cm,right=2.54cm,top=3.18cm,bottom=3.18cm}
\pagestyle{fancy}
\fancyhead[L]{\kaishu 强基数学001}
\fancyhead[C]{\kaishu 张卓立}
\fancyhead[R]{\kaishu 2204110786}

{
    \theoremstyle{plain}
    \theoremheaderfont{\normalfont\bfseries}
    \theorembodyfont{\kaishu}
    \newtheorem{exercise}{习题}
}

{
    \theoremstyle{nonumberplain}
    \theoremheaderfont{\bfseries}
    \theorembodyfont{\normalfont}
    \newtheorem{solution}{解.}
}

{
    \theoremstyle{nonumberplain}
    \theoremheaderfont{\bfseries}
    \theorembodyfont{\normalfont}
    \theoremsymbol{\ensuremath{\blacksquare}}
    \newtheorem{proof}{证明.}
}

\newcommand{\dif}{\mathrm{d}}
\newcommand{\differ}{\backslash}
\newcommand{\ptl}{\partial}
\newcommand{\R}{\mathbb{R}}
\newcommand{\N}{\mathbb{N}}
\newcommand{\C}{\mathbb{C}}
\newcommand{\Z}{\mathbb{Z}}
\newcommand{\sX}{\mathscr{X}}
\newcommand{\cl}{\mathcal{l}}
\renewcommand{\phi}{\varphi}
\renewcommand{\epsilon}{\varepsilon}
\renewcommand{\emptyset}{\varnothing}
\newcommand{\abs}[1]{\left\vert#1\right\vert}
\newcommand{\norm}[1]{\left\Vert#1\right\Vert}
\renewcommand{\liminf}{\varliminf}
\renewcommand{\limsup}{\varlimsup}
\renewcommand{\ae}{\ \mathrm{a.e.}\ }
\newcommand{\diam}{\mathrm{diam\ }}
\newcommand{\toinmeasure}[1]{\underset{#1}{\Rightarrow}}



\begin{document}
    
    \begin{center}
        \bfseries\LARGE
        泛函分析作业
    \end{center}

    \begin{exercise}
        \label{ex:1}
        考虑度量空间$\cl^2$,证明:$A:=\{\xi=\{x_n\}\in \cl^2:n\abs{x_n}\leqslant 1\}$是$\cl^2$中的紧集.
    \end{exercise}

    \begin{proof}
        设$\{\xi_k\}\subset A$,$\xi_k=\{x_n^{(k)}\}$,且有极限$\xi=\{x_n\}$,即$\sum_{n=1}^\infty \abs{x_n^{(k)}-x_n}\to 0,n\to \infty$,
        那么$x_n^{(k)}\to x_n,n\to\infty$,即$n\abs{x_n}\leqslant 1$,$\xi\in A$,即$A$是闭集.

        又对任意的$\epsilon>0$,存在$N$,使得$\sum_{n>N}\frac{1}{n^2}<\epsilon$.令$B=\{(x_1,x_2,\cdots,x_N,0,\cdots):\{x_n\}\in A\},B^\ast=\{
        (x_1,\cdots,x_N):\{x_n\}\in A\}\subset \R^N$.又因为$\abs{x_k}\leqslant\frac{1}{k}$,那么$B^\ast$有界.存在$\eta_1,\cdots,\eta_{n_\epsilon}\in B$
        对任意的$\xi\in B$,存在$j$,使得$\xi\in B(\eta_j,\epsilon)$,又因为$B$的定义可知,
        \begin{equation*}
            A\subset \bigcup_{\xi\in B}B(\xi,\epsilon),
        \end{equation*}
        即对任意的$x\in A$,存在$\xi\in B$,$\rho(x,\xi)<\epsilon$,又存在$\eta_j$,$\rho(\xi,\eta_j)<\epsilon$,则$\rho(x,\eta_j)<2\epsilon$,则
        \begin{equation*}
            A\subset \bigcup_{j=1}^{n_\epsilon}B(n_j,2\epsilon).
        \end{equation*}
        又$\cl^2$空间完备,在$\cl^2$中任意一个球中的点列总存在收敛子列,若有一个球$B(\xi_0,r_0)$,\\$B(\xi_0,r_0)\cap A\subset B(\xi_0,r_0)$不能被有限个以$A$中的点为球心,
        $r_0$为半径的球覆盖,那么存在点列$\{y_n\}$其中任意点的距离大于$r_0$,则它没有收敛子列,矛盾.则$A$有有穷的$\epsilon$-网,即$A$完全有界,那么$A$
        列紧,又$A$是闭集,则$A$自列紧,即$A$是紧集.
    \end{proof}

    \begin{exercise}
        \label{ex:2}
        用完备度量空间的充要条件证明Banach不动点定理.
    \end{exercise}

    \begin{proof}
        令$A_n=\{x\in X:\rho(x,Tx)\leqslant\frac{1}{n}\}$,则$A_n$递降.

        设$\{x_k\}$是$A_n$中的收敛点列,设$x_k\to x$,那么$\rho(x_n,Tx_n)\to\rho(x,Tx)\leqslant\frac{1}{n}$,即$A_n$是闭集.

        若存在$n_0$,$A_{n_0}=\emptyset$,即对任意的$x\in X$,$\rho(x,Tx)>\frac{1}{n_0}$.又因为$T$是压缩映射,存在$0<\alpha<1$,使得$\rho(Tx,Ty)\leqslant\alpha\rho(x,y)$,
        $\rho(T^kx,T^{k+1}x)\leqslant\alpha^k\rho(x,Tx)\to 0,k\to \infty$,则$A_{n_0}\neq\emptyset$,矛盾.则$\{A_n\}$均非空.

        下证$\diam A_n\to 0,n\to\infty$.
        \begin{align*}
            \diam A_n=\sup_{x,y\in A_n}&<\rho(x_0,y_0)+\frac{1}{n}\\
            &\leqslant \rho(x_0,Tx_0)+\rho(Tx_0,Ty_0)+\rho(Ty_0,y_0)+\frac{1}{n}\\
            &\leqslant \frac{1}{n}+\alpha\rho(x_0,y_0)+\frac{1}{n}+\frac{1}{n}\\
            &\leqslant \frac{3}{n}+\alpha\diam A_n.
        \end{align*}
        即$\diam A_n\leqslant\frac{3}{(1-\alpha)n}\to 0,n\to\infty$.

        存在$x\in X$,$\bigcap_{n=1}^\infty A_n=\{x\}$,即存在唯一的$x$使得$Tx=x$.
    \end{proof}
    
    \begin{exercise}
        \label{ex:3}
        设$K(\cdot,\cdot)\in L^2([a,b]\times [a,b])$,对于$f\in L^2[a,b]$,证明当$\lambda$充分小时,
        \begin{equation*}
            x(t)=f(t)+\lambda\int_a^bK(t,s)x(s)\dif s
        \end{equation*}
        在$L^2[a,b]$中存在唯一解.
    \end{exercise}

    \begin{proof}
        令$Tx=f(t)+\lambda\int_a^bK(t,s)x(s)\dif s$,那么$T$是从$L^2[a,b]$到$L^2[a,b]$的映射.

        \begin{align*}
            \rho(Tx,Ty)&=\left(\int_a^b\left(\lambda\int_a^bK(t,s)(x(s)-y(S))\dif s\right)^2\dif t\right)^{1/2}\\
            &=\lambda\left(\int_a^b\left(\int_a^bK(t,s)(x(s)-y(s))\dif s\right)^2\dif t\right)^{1/2}\\
            &=\rho(x,y)\lambda\left(\int_a^b \frac{\left(\int_a^b K(t,s)(x(s)-y(s))\dif s\right)^2}{\int_a^b(x(s)-y(s))^2\dif s}\dif t\right)^{1/2}\\
            &\leqslant \rho(x,y)\lambda\left(\int_a^b\left(\int_a^bK^2(t,s)\dif s\right)\dif t\right)^{1/2}.
        \end{align*}
        令$\lambda<\frac{1}{\sqrt{\int_{[a,b]^2}K^2(t,s)\dif t\dif s}+1}$,则$\lambda\left(\int_{[a,b]^2}K^2(t,s)\dif t\dif s\right)^{1/2}<1$,即$T$是压缩映射,
        又$L^2[a,b]$完备,那么存在唯一解.
    \end{proof}

    \begin{exercise}
        \label{ex:4}
        设$K(\cdot,\cdot)\in C([a,b]\times [a,b])$,$f\in C[a,b]$,证明:$\forall x\in \R$,
        \begin{equation*}
            x(t)=f(t)+\lambda\int_a^tK(t,s)x(s)\dif s
        \end{equation*}
        在$C[a,b]$中存在唯一解.
    \end{exercise}

    \begin{proof}
        令$Tx=f(t)+\lambda\int_a^tK(t,s)x(s)\dif s$,那么$T$是从$C[a,b]$到自身的映射.

        又因为
        \begin{align*}
            T^nx-T^ny&=\lambda\int_a^tK(t,s_1)(T^{n-1}x-T^{n-1}y)\dif s_1\\
            &=\cdots\cdots\\
            &=\lambda^n\int\limits_{\substack{
                a\leqslant s_1\leqslant t\\
                a\leqslant s_2\leqslant s_1\\
                \cdots\\
                a\leqslant s_n\leqslant s_{n-1}
            }}K(t,s_1)K(s_1,s_2)\cdots,K(s_{n-1},s_n)(x(s_n)-y(s_n))\dif s_n\cdots\dif s_1.
        \end{align*}
        那么 
        \begin{equation*}
            \rho(T^nx,T^ny)\leqslant \lambda^n I\rho(x,y),
        \end{equation*}
        其中 
        \begin{align*}
            I&=\int\limits_{a\leqslant s_n\leqslant s_{n-1}\leqslant\cdots\leqslant s_1\leqslant t}\abs{K(t,s_1)K(s_1,s_2)\cdots K(s_{n-1},s_n)}\dif s_n\dif s_{n-1}\cdots\dif s_1\\
            &\leqslant\frac{M}{n!}\int_{[a,t]^n}\abs{K(s_1,s_2)\cdots K(s_{n-1},s_n)}\dif s_n\cdots \dif s_1,\quad(\abs{K(t,s)}\leqslant M,(t,s)\in[a,b]^2)\\
            &\leqslant \frac{M^n(b-a)^n}{n!}.
        \end{align*}

        即
        \begin{equation*}
            \rho(T^nx,T^ny)\leqslant \frac{\lambda^nM^n(b-a)^n}{n!}\rho(x,y),
        \end{equation*}
        当$n$充分大时,$\frac{\lambda^nM^n(b-a)^n}{n!}<1$,此时$T^n$是压缩映射,则存在唯一解.
    \end{proof}

    \begin{exercise}
        \label{ex:5}
        $(X,\norm{\cdot})$,若$\dim X<\infty$,则$X$是$B$空间.
    \end{exercise}

    \begin{proof}
        设$\{x_n\}$是Cauchy列,即$\norm{x_m-x_n}\to 0,m,n\to\infty$,设$\dim X=l<\infty$,$e_1,\cdots,e_l$是它的基,令$x_n=\sum_{i=1}^lx_i^{(n)}e_i$.
        那么存在$c>0$,使得 
        \begin{equation*}
            c\left(\sum_{i=1}^l\abs{x_i^{(m)}-x_i^{(n)}}^2\right)^{1/2}\leqslant \norm{\sum_{i=1}^l(x_i^{(m)}-x_i^{(n)})e_i}\to 0,
        \end{equation*}
        即$\sum_{i=1}^l\abs{x_i^{(m)}-x_i^{(n)}}^2\to 0$,那么$\{x_i^{(n)}\}$是Cauchy列,$\{x_i^{(n)}\}$收敛,即$\{x_n\}$收敛,$X$是$B$空间.
    \end{proof}

    \begin{exercise}
        \label{ex:6}
        $(X,\norm{\cdot})$,$X$的任何有限维子空间是闭的.
    \end{exercise}

    \begin{proof}
        设$(X_0,\norm{\cdot})$是$X$的有限维子空间,那么它也是线性赋范空间,又因为它是有限维的,则它是$B$空间,若它不是闭的,那么存在收敛列$\{x_n\}\subset X_0$,
        极限为$x\notin X_0$,即$\{x_n\}$在$X_0$内不收敛,又因为它是$X_0$内的Cauchy列,与它是$B$空间矛盾.则$X_0$是闭的.
    \end{proof}

    \begin{exercise}
        \label{ex:7}
        $(X,\norm{\cdot})$.$\dim X<\infty$,$X$中的有界集是列紧集.
    \end{exercise}

    \begin{proof}
        设$\dim X=n<\infty$,$e_1,\cdots,e_n$是一组基,$A\subset X$有界.对任意的$x\in A$,$x=\sum_{i=1}^nx_ie_i$,那么存在$c>0,M>0$,使得 
        \begin{equation*}
            c\left(\sum_{i=1}^n\abs{x_i}^2\right)^{1/2}\leqslant\norm{x}\leqslant M.
        \end{equation*}
        即$(x_1,\cdots,x_n)$在$\R^n$中有界,那么若$\{x_n\}$是$A$中的点列,它们在基$e_1,\cdots,e_n$下的坐标在$\R^n$均位于球$B(0,M/c)$中,即存在
        收敛子列,那么对应$\{x_n\}$也存在收敛子列,$A$列紧.
    \end{proof}

    \begin{exercise}
        \label{ex:8}
        $X:=\{f\in C[0,1]:f(0)=0\}$,$\norm{f}=\max_{t\in[0,1]}\abs{f(t)}$,$X$是Banach空间.$X_0:=\{f\in X:\int_0^1 f(t)\dif t=0\}$,证明:$\dim X_0=\infty$.
    \end{exercise}

    \begin{proof}
        假设$\dim X_0=n<\infty$,考虑如下函数:
        \begin{equation*}
            \sin 2\pi x,\sin 4\pi x,\cdots,\sin 2\pi(n+1)x.
        \end{equation*}
        对任意的$k$,$\sin 2\pi kx\in X_0$,且$\{\sin 2\pi kx\}_{k=1}^{n+1}$线性无关,则$\dim X_0\geqslant n+1>n$,矛盾,则$\dim X_0=\infty$.
    \end{proof}

\end{document}