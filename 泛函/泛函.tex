\documentclass[a4paper,oneside,12pt]{ctexart}
\usepackage{enumerate,geometry,graphicx,bm,mathrsfs,xcolor,varwidth,framed,amsfonts,amssymb,indentfirst,fancyhdr}
\usepackage[colorlinks,linkcolor=red,anchorcolor=blue,citecolor=blue,urlcolor=blue]{hyperref}
\usepackage[thmmarks,hyperref]{ntheorem}
\usepackage{amsmath}

\setlength{\headheight}{15pt}
\allowdisplaybreaks[4]
\linespread{1.5}
\geometry{centering,left=2.54cm,right=2.54cm,top=3.18cm,bottom=3.18cm}
\pagestyle{fancy}
\fancyhead[L]{\kaishu 强基数学001}
\fancyhead[C]{\kaishu 张卓立}
\fancyhead[R]{\kaishu 2204110786}

{
    \theoremstyle{plain}
    \theoremheaderfont{\normalfont\bfseries}
    \theorembodyfont{\kaishu}
    \newtheorem{exercise}{习题}
}

{
    \theoremstyle{nonumberplain}
    \theoremheaderfont{\bfseries}
    \theorembodyfont{\normalfont}
    \newtheorem{solution}{解.}
}

{
    \theoremstyle{nonumberplain}
    \theoremheaderfont{\bfseries}
    \theorembodyfont{\normalfont}
    \theoremsymbol{\ensuremath{\blacksquare}}
    \newtheorem{proof}{证明.}
}

\newcommand{\dif}{\mathrm{d}}
\newcommand{\differ}{\backslash}
\newcommand{\ptl}{\partial}
\newcommand{\R}{\mathbb{R}}
\newcommand{\N}{\mathbb{N}}
\newcommand{\C}{\mathbb{C}}
\newcommand{\Z}{\mathbb{Z}}
\renewcommand{\phi}{\varphi}
\renewcommand{\epsilon}{\varepsilon}
\renewcommand{\emptyset}{\varnothing}
\newcommand{\abs}[1]{\left\vert#1\right\vert}
\newcommand{\norm}[1]{\left\Vert#1\right\Vert}
\renewcommand{\liminf}{\varliminf}
\renewcommand{\limsup}{\varlimsup}
\renewcommand{\ae}{\ \mathrm{a.e.}\ }
\newcommand{\diam}{\mathrm{diam\ }}
\newcommand{\toinmeasure}[1]{\underset{#1}{\Rightarrow}}



\begin{document}
    
\begin{center}
    \bfseries\LARGE
    泛函分析作业
\end{center}

\begin{exercise}
    \label{ex:1}
    考虑空间$C[a,b]$,令$\rho_1(x,y)=\max_{t\in[a,b]}\abs{x(t)-y(t)},\rho_2=\int_a^b\abs{x(t)-y(t)}\dif t$.证明:$(C[a,b],\rho_1)$是完备的
    度量空间,$(C[a,b],\rho_2)$不是完备的度量空间.
\end{exercise}

\begin{proof}
    先证明$(C[a,b],\rho_1)$是完备的度量空间.

    1. $\rho_1\geqslant 0$,且$\rho_1(x,y)=0\Leftrightarrow \abs{x(t)-y(t)}=0,\forall t\in[a,b]\Leftrightarrow x=y$.$\rho_1$满足正定性.

    2. $\rho_1(x,y)=\rho_1(y,x)$,$\rho_1$满足对称性.

    3. \begin{align*}
        \rho_1(x,z)&=\max_{t\in[a,b]}\abs{x(t)-z(t)}\\
        &\leqslant \max_{t\in[a,b]}(\abs{x(t)-y(t)}+\abs{y(t)-z(t)})\\
        &\leqslant \max_{t\in[a,b]} \abs{x(t)-y(t)}+\max_{t\in[a,b]}\abs{y(t)-z(t)}\\
        &=\rho_1(x,y)+\rho_1(y,z).
    \end{align*}
    $\rho_1$满足三角不等式.$(C[a,b],\rho_1)$是度量空间.

    下面证明$(C[a,b],\rho_1)$完备.设\{$x_n(t)\}\subseteq C[a,b]$是Cauchy列,那么对任意的$\epsilon>0$,存在$N$,对任意的$m,n>N$,
    $\rho_1(x_m,x_n)=\max_{t\in[a,b]}\abs{x_m(t)-x_n(t)}<\epsilon$,则$\{x_n(t)\}$一致收敛,令极限函数是$x(t)$,则$x(t)\in C[a,b]$,那么 
    $\rho_1(x_n,x)=\max_{t\in[a,b]}\abs{x_n(t)-x_(t)}\to 0,\quad n\to\infty$.$(C[a,b],\rho_1)$完备.

    再证明$(C[a,b],\rho_2)$不是完备的度量空间.

    1. $\rho_2(x,y)\geqslant 0$,且因为$x(t),y(t)$连续,$\rho_2(x,y)=\int_a^b\abs{x(t)-y(t)}\dif t=0\Leftrightarrow \\\abs{x(t)-y(t)}
    \equiv 0\Leftrightarrow x=y$.$\rho_2$满足正定性.

    2. $\rho_2(x,y)=\rho_2(y,x)$,$\rho_2$满足对称性.

    3. \begin{align*}
        \rho_2(x,z)&=\int_a^b\abs{x(t)-z(t)}\dif t\\
        &\leqslant \int_a^b(\abs{x(t)-y(t)}+\abs{y(t)-z(t)})\dif t\\
        &=\rho_2(x,y)+\rho_2(y,z).
    \end{align*}
    $\rho_2$满足三角不等式.$(C[a,b],\rho_2)$是度量空间.

    下面说明$(C[a,b],\rho_2)$不完备.反例:不妨令$a=0,b=1$,再令
    $$x_n=\begin{cases}
        -nx+1 & 0\leqslant x\leqslant 1/n\\
        0 & 1/n<x\leqslant 1.
    \end{cases}$$
    那么$x_n\in C[0,1]$,且$\rho_2(x_m,x_n)\to 0,\quad m,n\to \infty$,但是令
    $$x=\lim_{n\to\infty}x_n=\begin{cases}
        1 & x=0\\
        0 & 0<x\leqslant 1,
    \end{cases}$$
    $x\notin C[0,1]$,说明这个度量空间不完备.
\end{proof}

\begin{exercise}
    \label{ex:2}
    令$\rho(x,y)=\frac{\abs{x-y}}{1+\abs{x-y}}$,证明$(\R,\rho)$是完备的度量空间.
\end{exercise}

\begin{proof}
    1. $\rho(x,y)\geqslant 0$,且$\rho(x,y)=0\Leftrightarrow \abs{x-y}=0\Leftrightarrow x=y$.

    2. $\rho(x,y)=\rho(y,x)$.

    3. 注意到$\frac{x}{1+x}$当$x\geqslant 0$时是单调递增函数.\begin{align*}
        \rho(x,y)+\rho(y,z)&=\frac{\abs{x-y}}{1+\abs{x-y}}+\frac{\abs{y-z}}{1+\abs{y-z}}\\
        &=\frac{\abs{x-y}+\abs{y-z}+2\abs{x-y}\abs{y-z}}{1+\abs{x-y}\abs{y-z}+\abs{x-y}+\abs{y-z}}\\
        &\geqslant \frac{\abs{x-y}+\abs{y-z}+\abs{x-y}\abs{y-z}}{1+\abs{x-y}\abs{y-z}+\abs{x-y}+\abs{y-z}}\\
        &\geqslant \frac{\abs{x-y}+\abs{y-z}}{1+\abs{x-y}+\abs{y-z}}\\
        &\geqslant\frac{\abs{x-z}}{1+\abs{x-z}}=\rho(x,z).\qquad (\abs{x-y}+\abs{y-z}\geqslant \abs{x-z})
    \end{align*}
    则$(\R,\rho)$是度量空间.

    下面证明它完备.

    设$\{x_n\}$是Cauchy列,那么对任意的$\epsilon$,其中$0<\epsilon<1/2$,存在$N$对任意的$m,n>N$,$\frac{\abs{x_n-x_n}}{1+\abs{x_m-x_n}}<\epsilon$,
    那么$\abs{x_m-x_n}<\frac{\epsilon}{1-\epsilon}<2\epsilon$,则存在$x\in\R$,$\{x_n\}$收敛于$x$,$\rho(x_n,x)\leqslant \abs{x_n-x}\to 0,\quad n\to\infty$.
\end{proof}

\begin{exercise}
    \label{ex:3}
    $S[a,b]$表示$[a,b]$上几乎处处有界的可测函数全体.$\rho(f,g)=\int_a^b\frac{\abs{f-g}}{1+\abs{f-g}}\dif \mu$,证明$(S[a,b],\rho)$是完备的度量空间.
\end{exercise}

\begin{proof}
    1. $\rho(f,g)\geqslant 0$,$\rho(f,g)=0\Leftrightarrow \int_a^b\frac{\abs{f-g}}{1+\abs{f-g}}\dif \mu=0\Leftrightarrow f=g \ae x\in[a,b]$.

    2.$\rho(f,g)=\rho(g,f)$

    3.与题\eqref{ex:2}中证明三角不等式的过程类似,可以得到$\frac{\abs{f-g}}{1+\abs{f-g}}+\frac{\abs{g-h}}{1+\abs{g-h}}\geqslant\frac{\abs{f-h}}{1+\abs{f-h}}$.
    那么$\rho(f,h)\leqslant\rho(f,g)+\rho(g,h)$.综上,$(S[a,b],\rho)$是度量空间.

    下面证明$(S[a,b],\rho)$完备.

    设$\{f_n\}\subseteq S[a,b]$是Cauchy列,即$\rho(f_m,f_n)\to 0,\quad m,n\to\infty$,则$\mu(\abs{f_m-f_n}>\delta)\to 0,\quad m,n\to\infty$,那么存在
    $f$可测,$f_n\toinmeasure{\mu} f$.存在子列$\{f_{n_k}\}$,使得$\mu(\abs{f_{n_k}-f}>1)<\frac{1}{2^k}$,令$A_k=\{x:\abs{f_{n_k}(x)-f(x)}>1\}$,则
    $\mu(A_{k})<\frac{1}{2^k}$,且$\mu(\limsup_{k\to\infty} A_k)=\mu(\bigcap_{m=1}^\infty\bigcup_{k=m}^\infty A_k)\leqslant \frac{1}{2^{m-1}},\ \forall m\geqslant 1$.
    则$\mu(\limsup_{k\to\infty} A_k)=0$.而且存在$k_0$,使任意的$x\in[a,b]\differ\limsup_{k\to\infty} A_k$有$\abs{f_{n_{k_0}}(x)-f(x)}\leqslant 1$,又因为
    $f_{n_{k_0}}$几乎处处有界,那么$f\in S[a,b]$.

    $\rho(f_n,f)\leqslant\int_a^b\abs{f_n-f}\dif \mu$,又$\mu([a,b])<\infty$,令$E_0(n)=\{\abs{f_n-f}\geqslant 1\},E_1(n)=\left\{\frac{1}{2}\leqslant\abs{f_n-f}<1\right\}$,
    $E_2(n)=\{\frac{1}{2^2}\leqslant\abs{f_n-f}<\frac{1}{2}\},\cdots$,那么 
    \begin{align*}
        \int_a^b\abs{f_n-f}\dif \mu&=\sum_{k=0}^\infty \int_{E_k(n)}\abs{f_n-f}\dif \mu\\
        &<\int_{E_0(n)}\abs{f_n-f}\dif \mu+\sum_{k=1}^\infty\frac{1}{2^{k-1}}\mu(E_k(n)).
    \end{align*}
一方面,$\int_{E_0(n)}\abs{f_n-f}\dif \mu\to 0\quad n\to\infty$,另一方面,因为$\mu(E_K(n))\leqslant\mu([a,b])<\infty$,那么级数$\sum_{k=1}^\infty \frac{1}{2^{k-1}}\mu(E_k(n))$
一致收敛,$\lim_{n\to\infty}\sum_{k=1}^\infty \frac{1}{2^{k-1}}\mu(E_k(n))=0$.即\\$\lim_{n\to\infty}\int_a^b\abs{f_n-f}\dif\mu=0$.$(S[a,b],\rho)$
完备.
\end{proof}

\begin{exercise}
    \label{ex:4}
    $1\leqslant p< \infty$,令$\rho(f,g)=\left(\int_a^b\abs{f-g}^p\dif x\right)^{1/p}$,证明$(L^p[a,b],\rho)$是完备的度量空间.
\end{exercise}

\begin{proof}
    1. $\rho(x,y)\geqslant 0$,当$1\leqslant p<\infty$时,$\rho(x,y)=0\Leftrightarrow \int_a^b\abs{f-g}^p\dif x=0\Leftrightarrow f=g\ae$.

    2.对任意的$1\leqslant p< \infty$,$\rho(f,g)=\rho(g,f)$.

    3.当$p=1$时显然成立.当$1<p<\infty$,令$\frac{1}{p}+\frac{1}{q}=1$即$p-1=\frac{p}{q}$,那么 
    \begin{align*}
        \abs{f-h}^p&=\abs{f-g+g-h}^p\\
        &=\abs{f-g+g-h}\abs{f-h}^{p/q}\\
        &\leqslant \abs{f-g}\abs{f-h}^{p/q}+\abs{g-h}\abs{f-h}^{p/q}.
    \end{align*}
    则 
    \begin{align*}
        \int_a^b\abs{f-h}^p\dif x&\leqslant \int_a^b\abs{f-g}\abs{f-g+g-h}^{p/q}\dif x+\int_a^b\abs{g-h}\abs{f-g+g-h}^{p/q}\dif x\\
        &\leqslant \left(\int_a^b\abs{f-g}^p\dif x\right)^{1/p}\left(\int_a^b\abs{f-h}^p\right)^{1/q}\\
        &\quad +\left(\int_a^b\abs{g-h}^p\dif x\right)^{1/p}\left(\int_a^b\abs{f-h}^p\right)^{1/q}\\
        &=\left[\left(\int_a^b\abs{f-g}^p\dif x\right)^{1/p}+\left(\int_a^b\abs{g-h}^p\dif x\right)^{1/p}\right]\left(\int_a^b\abs{f-h}^p\dif x\right)^{1/q}.
    \end{align*}
    整理可得$\rho(f-h)\leqslant\rho(f-g)+\rho(g-h)$.

    下面证明它是完备的.

    设$\{f_n\}\subseteq L^p[a,b]$是Cauchy列.存在子列$\{f_{n_k}\},\rho(f_{n_k},f_{n_{k-1}})<\frac{1}{2^k}$.因为$k_{n_k}=f_{n_1}+\sum_{j=2}^k(f_{n_j}-f_{n_{j-1}})$,
    那么令
    \begin{equation*}
        \abs{f_{n_k}}\leqslant\abs{f_{n_k}}+\sum_{j=2}^k\abs{f_{n_k}-f_{n_{j-1}}}=g_k(x).
    \end{equation*}

    根据三角不等式,
    \begin{align*}
        \left(\int_a^b\abs{g_k(x)}^p\dif x\right)^{1/p}&=\left(\int_a^b\abs{\abs{f_{n_1}}+\sum_{j=2}^k\abs{f_{n_j}-f_{n_{j-1}}}}^p\dif x\right)^{1/p}\\
        &\leqslant \left(\int_a^b\abs{f_{n_1}}^p\dif x\right)^{1/p}+\left(\int_a^b\abs{\sum_{j=2}^k\abs{f_{n_j}-f_{n_{j-1}}}}^p\dif x\right)^{1/p}\\
        &\leqslant\cdots\leqslant \left(\int_a^b\abs{f_{n_1}}^p\dif x\right)^{1/p}+\sum_{j=2}^k\left(\int_a^b\abs{f_{n_j}-f_{n_{j-1}}}^p\dif x\right)^{1/p}\\
        &\leqslant \left(\int_a^b\abs{f_{n_1}}^p\dif x\right)^{1/p}+\sum_{j=2}^k\frac{1}{2^j}\\
        &<\left(\int_a^b\abs{f_{n_1}}^p\dif x\right)^{1/p}+\frac{1}{2}.
    \end{align*}

    令$g(x)=\abs{f_{n_1}}+\sum_{j=2}^\infty\abs{f_{n_j}-f_{n_{j-1}}},f(x)=f_{n_1}+\sum_{j=2}^\infty(f_{n_j}-f_{n_{j-1}})$,$\{g_k(x)\}$单调递增,则$g(x)\in L^p[a,b]$,
    又因为$\abs{f}\leqslant g$,那么$f\in L^p[a,b]$.

    又
    \begin{align*}
        \rho(f_n,f)&=\left(\int_a^b\abs{f_{n_k}-f}^p\dif x\right)^{1/p}\\
        &\leqslant \left(\int_a^b\abs{f_{n_k}}^p\dif x\right)^{1/p}+\left(\int_a^b\abs{f}^p\dif x\right)^{1/p}\\
        &\leqslant 2\left(\int_a^b\abs{g(x)}^p\dif x\right)^{1/p}.
    \end{align*}
    那么$\lim_{k\to\infty}\rho(f_{n_k},f)=\left(\int_a^b \lim_{k\to\infty}\abs{f_{n_k}-f}^p\dif x\right)^{1/p}=0$.
    则$\rho(f_n,f)\to 0\quad n\to\infty$.$(L^p[a,b],\rho)$是完备的.
\end{proof}

\begin{exercise}
    \label{ex:5}
    $(X,\rho)$是度量空间,$\forall\{x_n\}\subset A$,那么由$x_n\to x$得到$x\in A$等价于$A$是$X$中的闭集.
\end{exercise}

\begin{proof}
    ``$\Leftarrow$''.若$A$是$X$中的闭集,那么对$A$中的任意收敛列,它的极限一定在$A$中.

    ``$\Rightarrow$''.若对$A$中任意的收敛列$\{x_n\}$,它的极限在$A$中,那么$A'\subset A$,即$\overline{A}=A$,$A$是闭集.
\end{proof}

\begin{exercise}
    \label{ex:6}
    $(X,\rho)$是度量空间,$A\subset X$,证明$\diam A=\diam \overline{A}$.
\end{exercise}

\begin{proof}
    一方面,$\diam A=\sup_{x,y\in A}\rho(x,y)\leqslant \sup_{x,y\in\overline{A}}\rho(x,y)=\diam \overline{A}$.

    另一方面,对任意的$\epsilon>0$,存在$x_0,y_0\in\overline{A}$,使得
    \begin{equation*}
        \diam \overline{A}<\rho(x_0,y_0)+\epsilon.
    \end{equation*}
    又因为$x_0,y_0\in\overline{A}$,那么存在$x_1,y_1\in A$,使得$\rho(x_1,x_0),\rho(y_1,y_0)<\epsilon$,那么 
    \begin{align*}
        \rho(x_0,y_0)&\leqslant \rho(x_0,x_1)+\rho(x_1+y_1)+\rho(y_1,y_0)\\
        &<\rho(x_1,y_1)+2\epsilon.
    \end{align*}
    则 
    \begin{equation*}
        \diam \overline{A}<\rho(x_1,y_1)+3\epsilon\leqslant \diam A+3\epsilon.
    \end{equation*}
    根据$\epsilon$的任意性,$\diam\overline{A}\leqslant\diam A$,综上,$\diam\overline{A}=\diam A$.
\end{proof}

\begin{exercise}
    \label{ex:7}
    $(X,\rho)$是度量空间,设$E\subset X$,则$E$是疏集$\Leftrightarrow$ 对任意的$\overline{B(x,r)}$,必存在开球$B(x',r')\subset B(x,r)$,
    使得$\overline{B(x',r')}\cap E=\emptyset$.
\end{exercise}

\begin{proof}
    ``$\Rightarrow$''.若存在$B(x,r)$,使得对任意的$B(x',r')\subset B(x,r)$,$\overline{B(x',r')}\cap E\neq \emptyset$,那么$E$在$B(x,r)$
    稠密,矛盾.

    ``$\Leftarrow$''.对任意的$B(x,r)$,都存在$B(x',r')\subset B(x,r)$,使$\overline{B(x',r')}\cap E=\emptyset$,那么$E$无内点,$E$是疏集.
\end{proof}

\end{document}