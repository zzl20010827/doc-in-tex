\documentclass[12pt,a4paper,oneside]{ctexart}
\usepackage{caption,graphicx,amsmath,amsthm,amsfonts,amssymb,bm,mathrsfs,extarrows,
indentfirst,setspace,textcomp,xcolor,color,colortbl,booktabs,float,subcaption,
fancyhdr}
\usepackage[centering]{geometry}
\usepackage{array,blkarray}
\usepackage{tikz,tikz-3dplot,asymptote}
\usepackage{listings}

\geometry{left=2.54cm,right=2.54cm,top=3.18cm,bottom=3.18cm}
\usetikzlibrary{arrows.meta}
\setlength{\headheight}{15pt}
\everymath{\displaystyle}
\newfontfamily\consolas{Consolas}
\definecolor{lightgray}{gray}{0.5}
\linespread{1.5}

\newcommand{\dif}{\mathrm{d}}
\newcommand{\differ}{\backslash}
\newcommand{\ptl}{\partial}
\newcommand{\R}{\mathbb{R}}
\newcommand{\N}{\mathbb{N}}
\newcommand{\C}{\mathbb{C}}
\newcommand{\D}{\mathbb{D}}
\newcommand{\Z}{\mathbb{Z}}
\renewcommand{\phi}{\varphi}
\renewcommand{\epsilon}{\varepsilon}
\newcommand{\abs}[1]{\left\vert#1\right\vert}
\newcommand{\norm}[1]{\left\Vert#1\right\Vert}

\title{思路}
\date{}
\author{}

\begin{document}

  \maketitle

  \textbf{第二问.}

\end{document}



思路可能还会再改.

\section{第一问}

首先是定量描述传播过程,我觉得描述传播过程最好的变量就是转发量,因为转发量描述了截至目前被转发的次数.所以定量描述传播过程用转发量随时间的变化来刻画.另外
还有阅读量、评论量、点赞量.

接下来是影响因素,首先对传播过程造成影响的就是转发量,其次分别是阅读量、评论量、点赞量,因为这些变量刻画了已参与用户的观点的热度,所以会对之后的用户产
生激励作用,从而促进话题的传播.(《基于自激点过程的网络热点话题传播模型》)

然后是如何刻画最终的观点.我的想法第一是再统计点踩量,用$\frac{\mbox{点赞量}}{\mbox{点赞量$+$点踩量}}$来刻画最终观点是中立共识还是观点分化,数值越是接近$\frac{1}{2}$,越是接近观点分化;
数值越是接近0或是1,越是观点接近相同,也就是越接近中立共识.另外,还要收集网站话题的内容和评论,然后对这些评论的观点分析,统计支持量和反对量,计算$\frac{\mbox{支持量}}{\mbox{支持量$+$反对量}}$,二者求平均,
因为$\frac{\mbox{点赞量}}{\mbox{点赞量$+$点踩量}}$刻画太粗糙,而第二种$\frac{\mbox{支持量}}{\mbox{支持量$+$反对量}}$有可能出现错误.得到的平均值暂时记成$f(t)$,$t$表示话题编号,$f$从0到1表示越来越受欢迎,观点从大多数反对到
极化到大多数支持.

\section{第二问}

首先是中立共识和观点分化的形成机制.中立共识和观点分化程度的定量描述已经在第一问说明,只需要确定哪些因素可能对最终的观点造成影响.我觉得首先是话题本身的内容不同,争议性不同,对最后的观点是中立还是极化会造成影响,现在的问题是怎么衡量
话题的内容,初步想法是话题的分类,比如说社会热点,国内政治,国际政治,还有是正面的还是负面的.然后是传播的媒介:微博,贴吧,自媒体,官方媒体,意见领袖,媒介不同受众可能也不同.然后是用户的分类:年龄段,性别,受教育程度.然后用主成分分析得到的主成分中各个影响因素的
权重与第一问量化的观点情况比较来说明形成机制.

接下来是``尖叫效应''``回声室效应''和``信息茧房''的形成机制. ``尖叫效应''``回声室效应''和``信息茧房''的解释:
\begin{quotation}
  \textbf{尖叫效应:}``尖叫效应''在资讯传播中也得以显著体现,通过非法抓取、剪拼改编的惊悚、恶搞、色情等低俗内容,往往能迅速引发人们的大量关注,无论是从满足人们的猎奇心理,还是引发人们的指责批评,传播者都能从中获取高额的流量和点击率.

  \textbf{回声室效应:}在媒体上是指在一个相对封闭的环境上,一些意见相近的声音不断重复,并以夸张或其他扭曲形式重复,令处于相对封闭环境中的大多数人认为这些扭曲的故事就是事实的全部.简言之,即信息或想法在封闭的小圈子里得到加强.

  \textbf{信息茧房:}信息茧房是指人们关注的信息领域会习惯性地被自己的兴趣所引导,从而将自己的生活桎梏于像蚕茧一般的``茧房''中的现象.
\end{quotation}

所以``尖叫效应''可能与话题内容,传播媒介有关,``回声室效应''可能与用户的那些分类有关,然后分别主成分分析得到各个因素的情况说明形成机制.又因为尖叫效应和
回声室效应都可能导致信息茧房,所以取主成分权重绝对值最大的来作为影响因素不知道行不行.

接下来是话题的吸引度、用户的活跃度、用户心理、不同用户间的相互影响、平台推荐算法等因素对形成这些现象的影响.吸引度可以用点击量定量表示,用户活跃度可以用评论数/点击量定量表示,用户心理用点赞量表示,用户之间的影响用
转发量表示,另外再统计平台推荐算法,分别画出$f$关于这些量的示意图来看对信息茧房的影响.

\section{第三问\&第四问}
那就根据第二问的结果具体说说.